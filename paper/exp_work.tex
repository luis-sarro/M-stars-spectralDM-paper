
% Comment. I move the description of the IRTF data to the point where these are actually used.

{ The objective addressed in this Section is to develop a procedure to
  identify spectral bands that yield good temperature, gravity and
  metallicity diagnostics. Given the lack of a calibration set of
  observed spectra with homogeneous coverage of the space of physical
  parameters, we turn to synthetic libraries of spectra. The atomic or
  molecular line/band parameters could in principle indicate the
  spectral features that are more sensitive to changes in the physical
  parameters. The suitability of spectral features as diagnostics
  ofthe stellar atmospheric properties depends not only on the
  individual behaviour of each line/band, but also on the relative
  properties of neighbouring features in the same spectral region,
  that may overlap depending on the spectral resolution. Furthermore,
  good spectral diagnostics at a given signal-to-noise ratio (SNR) may
  show a severy degraded predictive power in the low SNR regime. In
  the following we adopt the BT-Settl library of synthetic spectra
  (\cite{2013MSAIS..24..128A}) as the framework where spectral
  diagnostics will be searched for. These synthetic spectra were
  pre-processed in several steps as described below.}

{ First, and in order to define good temperature diagnostics, spectra
between 2000 and 4200K in steps of 100 K were selected, with $\log(g)$
in the range between 4 and 6 dex (when g is expressed in cm/s$^{-2]$),
in steps of 0.5 dex. The metallicity of the representative spectra was
restricted to the set 0, 0.5 and -1 dex.  This yields a total set
size of 535 available spectra.

The spectral resolution was degraded to the IRTF resolution factor (R
~ 2000) by convolving with a Gaussian. Then, the spectra were trimmed
to produce valid segments between 8145.92 and 24106.85{\AA}, which is
the spectral range common to all M stars in the IRTF library. Finally,
all spectra were divided by the total integrated flux in this range in
order to factor out the stellar distance.}


{ In order to increase the density of examples in parameter space, we
  introduced interpolated spectra in the BT-Settl grid. Interpolation
  was obtained as a linear combination of spectra in the grid,
  weighted by the inverse square of the euclidean distance. {\bf Aqui,
  la distancia euclidea deberia calcularse en parametros normalizados,
  porque si no la temperatura domina la distancia. Fue asi?} We
  compared a set of interpolated spectra with those produced using the
  PHOENIX code (\cite{fuhrmeister2005phoenix}) to be sure that
  interpolation was a valid solution to infer new synthetic
  spectra. {\bf Yo aqui daría el RMSE de reconstruccion, mejor que la
  figura comp-gen-inter}

%(see Fig.~\ref{fig:comp_gen_inter}).  }

%\begin {figure}
% \begin{center}
% \includegraphics[width=6cm]{figs/intgrid4_gauss.pdf}
% \caption{Comparison between generated and interpollated spectrum}
% \label{fig:comp_gen_inter}
% \end{center}
%\end {figure}

{ A first interpolation stage allowed us to define a finer mesh step of
0.25 dex for both, $\log(g)$ and metallicity and 50K in temperature,
yielding a total 1329 spectra.  Then, a second interpolation stage
refined the grid down to 25 K in temperature and 0.125 dex in 
$\log(g)$, keeping the metallicity step at 0.25 dex and producing a
dataset with 25912 spectra.}

{In spite of these, and in order to keep their knowledge closer to the
original BT-Settl source, most of the analyses have been performed
with the original 535 spectra dataset.  }{\bf Habría que delimitar
exactamente donde se han utilizado 535 y dónde 25912. Si la mayoria
del analisis se ha realizado sobre 535, no se si tiene sentido incluir
la parte de interpolacion.}


{ In order to avoid selecting spectral features that are good
  predictors only in the unrealistic SNR=$\infty$ regime, the search
  for optimal diagnostics of the atmopheric paramters of M stars was
  carried out for three SNR values (10, 50 and $\infty$) by degrading
  the synthetic spectra with Gaussian noise of zero mean.  {\bf Quizás
  deberíamos citar el trabajo de Ana como in preparation} }

\subsection{Feature definition and selection}
\label{subsec:FD}
{ As mentioned in Sect. \ref{sec:intro}, it is well known the
difficulty in defining good spectral diagnostics for M stars in the
infrared.}

{The work in \cite{2013A&A...549A.129C} defined wavelength regions in
the I and K bands optimal for the disagnostic of physical parameters
based on the sensitivity exhibited by the flux emitted in these
segments to changes of the physical parameters. The sensitivity was
measured in terms of the derivative of the flux with respect to the
physical parameter. The approach adopted in this work was to define
the spectral features as those producing the best estimates of
$T_{eff}$, $\log(g)$ and the metallicity. The accuracy of the
estimates produced from a subset of features is established from
regression models described further below.

We consider the effective temperature as the dominant parameter
influencing changes in the stellar spectra (a strong feature) and
thus, it was estimated first, and then used as  in the regression models for the
gravity and metalicity.

The central contribution of this paper is the way to identify the most suitable
regions of spectrum (features) to be considered as signal providers 
but mainly the regions to be considered 
as candidates for being continuum for those signal regions.
It will be performed by means of artificial intelligence techniques.

These features consist of
a central bandpass covering the interesting lines and another bandpass
referring to the local continuum. Then, the feature can be written like 
Eq.~\eqref{eq:feature}.

\begin{equation}\label{eq:feature}
  F(i) =  \frac{ \int_{\lambda_{1s}(i)}^{\lambda_{2s}(i)} \left(f(\lambda)\right) d{\lambda}}
               { \int_{\lambda_{1c}(i)}^{\lambda_{2c}(i)} \left(f(\lambda)\right) d{\lambda}} 
               \quad \quad \forall i \in [1 \ldots N]
\end{equation}


where:
\begin{itemize}
 \item {N means the number of features to be considered, as decided by the researcher.}
 \item {$f(x)$ denotes the normalized flux spectra from the star.}
 \item {$(\lambda_{1s},\lambda_{2s})$ {\AA} accounts for the region of spectrum where the signal is considered. \quad \label{eq:cons1}}
 \item {$(\lambda_{1c},\lambda_{2c})$ {\AA} accounts for the region of spectrum where the continuum is considered.}
\end{itemize}

{
Now, the research question is how to identify 
$\{\lambda_{1s}(i),\lambda_{2s}(i), \lambda_{1c}(i),\lambda_{2c}(i)\}  \quad \forall i \in [1 \ldots N] $
in such a way they become useful to estimate the physical parameters.

Some constaints were estabished as designing of the selection process:

\begin{itemize}
 \item { $ \vert \lambda_{2k(i)} - \lambda_{1k(i)} \vert = 30 $ pixels of spectrum $ \quad \forall i \in [1 \ldots N]$ and $ k \in \{s,c\} $ .}
 \item { $ min ( \lambda_{1k(i)}, \lambda_{1k(j)} ) = 5 $ pixels of spectrum $ \quad \forall i,j \in [1 \ldots N] $ ; $ i \neq j $ and $ k \in \{s,c\} $ .}
 \item { $ \overline{\lambda_{2}(i)\lambda_{1}(i)}  \bigcap 
                      \overline{\lambda_{2c}(i)\lambda_{1c}(i)} = \emptyset \quad \forall i \in [1 \ldots N]$.}
\end{itemize}

which become a guarantee avoiding any overlap 
and a minimum size for both signal and continuum bandpasses.
}

{
Search for those features will depend on which specific physical parameter is 
under consideration but, the proposed methodology will look for those values 
trying to solve an optimization problem, which shall be the forecast capabilities
of one specific set of features, to be retained when it becomes bigger than 
a threshold.

To accomplish such optimization problem involving the selection of variable 
subsets, the use of the Genetic Algorithms technique was accepted. 
}

{ It was proposed to use the software tools R(\cite{R2013}).  There
  are different statistics to identify features that are
  differentially expressed between two or more groups of samples and
  then uses the most differentially expressed to construct a
  statistical model.  }

{ These methods have demonstrated to perform well, however, in some
  cases they can be ineffective regardless of the classification
  method used. An obvious conceptual limitation of univariate
  approaches is also the lack of consideration that features works in
  the contexts of interconnected pathways and therefore it is their
  behavior as a group that may be predictive of the phenotypic
  variables. Multivariate selection methods may seem to be more
  suitable for the analysis of data since variables are tested in
  combination to identify interactions between features. However, the
  extremely large number of models that can be constructed from
  different combination of thousands of features cannot be extensively
  evaluated using standard computational resources.  }

{ For the sake of simplicity let us define Genetic Algorithms (GAs)
  are variable search procedures that are based on the principle of
  evolution by natural selection. The procedure works by evolving sets
  of variables (chromosomes) that fit certain criteria from an initial
  random population via cycles of differential replication,
  recombination and mutation of the fittest chromosomes. The concept
  of using in-silico evolution for the solution of optimization
  problems has been introduced by John Holland in 1975
  (\cite{holland1975adaptation}). Although their application has been
  reasonably widespread (see Goldberg\textquoteright s book
  (\cite{goldberg1989genetic}), they became very popular only when
  sufficiently powerful computers became available.  }

{
The implementation of the GA follows the next steps:
\begin{itemize}
 \item [\textbf{Stage 1}:]{To produce the population of potential features (chromosomes).}
 \item [\textbf{Stage 2}:]{Each chromosome in the population is evaluated for its ability to
predict the group membership of each sample in the dataset (fitness function).}
 \item [\textbf{Stage 3}:]{Chromosome preselection, when a chromosome has 
 a score higher then a predefined value.}
 \item [\textbf{Stage 4}:]{The population of chromosomes is replicated. 
 Chromosomes with a higher fitness score will 
 generate a more numerous offspring.}
 \item [\textbf{Stage 5}:]{The genetic information contained in the replicated parent
chromosomes is combined through genetic crossover. Two randomly selected
parent chromosomes are used to create two new chromosomes.}
 \item [\textbf{Stage 6}:]{Mutations are then introduced in the chromosome randomly. 
 These mutations produce that new genes are used in chromosomes.
 Steps 5 and 6 are applied over the chromosomes establised at Step 4.}
  \item [\textbf{Stage 7}:]{This process is repeated from Stage 2 until 
  enough accuracy is obtained or the maximum of iterations is attained.}
\end{itemize}

The features were constructed as indicated above and named according
to the ordinal of the wavelength step both for signal and
continuum. The name includes also the number of ofset induced because
of the constraint \ref{eq:cons1}.  Polulation size was choosed as one
thousand individuals and accepted iterations were four thousand.
Three randomly started different repetions where produced bringing the
opportunity for enough variety and probabilities were established as
0.85 to corossover and 0.35 to mutation. The elitism was fixed to be
0.15.  Fitness for features were established as related to the Akaike
Criterion (-AIC) for linearity between the potential feature against
the physical parameter.  The most frequent and efficient features were
suggeswted as candidates to describe the behavior of specific physical
parameters.

From the implementation point of view a binarized codification was
selected in accordance to the naming convention and in order to speed
up the computation, a parallel implementation from a farm of fifteen
connected computers were used for each of the physical parameters.  }


{ The GA procedure provides us with a large collection of chromosomes.
  Although these are all potential solutions of the problem, it is not
  clear which one should be chosen for developing a model becoming for
  interpretation.  For this reason there is a need to develop a single
  model that is, to some extent, representative of the population. The
  simpler strategy to follow is to use the frequency of the chromosme
  in the population of chromosomes as criteria for inclusion in a
  forward selection strategy, however for this particular applciation,
  the choice was to include features based on their highest fitness.

After applying this technique the recommended features for temperature 
can be found in Table~\ref{tab:tab_NC_T}. 

\begin{table}
\begin{center}
\begin{tabular}{rrrrrrr}
  \hline
 & Signal From & Signal To & Continuum From & Continuum To & Fitness & Freq \\ 
  \hline
 Feature 1 & 8376.10 & 8433.91 & 9346.13 & 9403.92 & -6693.10 & 319 \\ 
  Feature 2 & 8385.99 & 8443.94 & 9346.13 & 9403.92 & -6700.15 &   6 \\ 
  Feature 3 & 8195.96 & 8254.03 & 9386.01 & 9444.05 & -6887.55 &  44 \\ 
  Feature 4 & 8186.06 & 8243.98 & 9235.98 & 9294.01 & -7056.23 &  19 \\ 
  Feature 5 & 8406.00 & 8464.07 & 9515.96 & 9574.13 & -7068.86 &  34 \\ 
%  Feature 6 & 9326.07 & 9384.15 & 8406.00 & 8464.07 & -7349.99 &  18 \\ 
%  Feature 7 & 8496.05 & 8554.06 & 9576.03 & 9634.04 & -7505.21 &  38 \\ 
%  Feature 8 & 9036.07 & 9094.04 & 9075.93 & 9133.98 & -7535.71 &  15 \\ 
%  Feature 9 & 9135.89 & 9193.92 & 9085.96 & 9144.03 & -7622.16 &  27 \\ 
%  Feature 10 & 9515.96 & 9574.13 & 8876.08 & 8934.03 & -7655.59 &   6 \\ 
%  Feature 11 & 8716.00 & 8773.99 & 9025.93 & 9084.07 & -7703.40 &  77 \\ 
%  Feature 12 & 9156.03 & 9214.07 & 8255.97 & 8314.06 & -7708.62 &  15 \\ 
%  Feature 13 & 8266.11 & 8324.03 & 8235.96 & 8294.04 & -7856.20 &  30 \\ 
%  Feature 14 & 8235.96 & 8294.04 & 8255.97 & 8314.06 & -7860.73 &  69 \\ 
%  Feature 15 & 8705.93 & 8763.97 & 8886.00 & 8943.99 & -7919.62 &  27 \\ 
%  Feature 16 & 8536.03 & 8594.06 & 8336.02 & 8394.08 & -7940.67 &  24 \\ 
%  Feature 17 & 8605.97 & 8663.96 & 8346.02 & 8404.06 & -7953.63 &  59 \\ 
%  Feature 18 & 8946.07 & 9004.01 & 8756.09 & 8814.06 & -8117.90 &  46 \\ 
%  Feature 19 & 9135.89 & 9193.92 & 9485.83 & 9544.11 & -8211.98 &  36 \\ 
%  Feature 20 & 8536.03 & 8594.06 & 8496.05 & 8554.06 & -8337.36 &  31 \\ 
   \hline
\end{tabular}
\caption {Recommended features and Continuum bandpass for predicting $ T_{eff} $ 
      by using BT\_Settl with SNR= $ {\infty} $ . 
      The Fitness and frequency of occurence are also included.} \label{tab:tab_NC_T} 
\end{center}
\end{table}

The authors have estimated the suggested features when theoretical BT\_Settl 
is noised with different SNR and following tables \ref{tab:tab_SNR10_T} 
and \ref{tab:tab_SNR50_T} sumarize the findings.


\begin{table}
\begin{center}
\begin{tabular}{rrrrrrr}
  \hline
 & Signal From & Signal To & Continuum From & Continuum To & Fitness & Freq \\ 
  \hline
Feature 1 & 8385.99 & 8443.94 & 9395.94 & 9454.03 & -6734.59 & 136 \\ 
  Feature 2 & 8186.06 & 8243.98 & 9536.15 & 9593.96 & -6857.65 &   7 \\ 
  Feature 3 & 8186.06 & 8243.98 & 9376.07 & 9433.92 & -6947.54 &   7 \\ 
  Feature 4 & 8286.01 & 8343.92 & 9206.05 & 9264.00 & -7123.10 &  10 \\ 
  Feature 5 & 8355.96 & 8414.03 & 9066.05 & 9124.05 & -7207.55 &  37 \\ 
 % Feature 6 & 9276.00 & 9333.87 & 8415.91 & 8473.96 & -7436.13 &  32 \\ 
 % Feature 7 & 8455.96 & 8513.93 & 9055.94 & 9114.07 & -7476.12 &  32 \\ 
 % Feature 8 & 8616.00 & 8673.98 & 9576.03 & 9634.04 & -7491.24 &   6 \\ 
 % Feature 9 & 8536.03 & 8594.06 & 9135.89 & 9193.92 & -7573.72 &  58 \\ 
 % Feature 10 & 9395.94 & 9454.03 & 9356.05 & 9414.08 & -7586.78 &  45 \\ 
 % Feature 11 & 9576.03 & 9634.04 & 9045.91 & 9103.99 & -7641.39 &  19 \\ 
 % Feature 12 & 9066.05 & 9124.05 & 8166.02 & 8224.12 & -7642.28 &  15 \\ 
 % Feature 13 & 9386.01 & 9444.05 & 9536.15 & 9593.96 & -7684.66 &   7 \\ 
 % Feature 14 & 8616.00 & 8673.98 & 8756.09 & 8814.06 & -7686.46 &  17 \\ 
 % Feature 15 & 9576.03 & 9634.04 & 8145.92 & 8204.03 & -7767.19 &  13 \\ 
 % Feature 16 & 9466.08 & 9523.82 & 9036.07 & 9094.04 & -7772.45 &  37 \\ 
 % Feature 17 & 9445.97 & 9504.01 & 9545.87 & 9604.02 & -7830.60 &  35 \\ 
 % Feature 18 & 8286.01 & 8343.92 & 8486.02 & 8544.05 & -7863.15 &  49 \\ 
 % Feature 19 & 9186.03 & 9244.04 & 9135.89 & 9193.92 & -7884.30 &  13 \\ 
 % Feature 20 & 9306.03 & 9363.93 & 8745.93 & 8803.93 & -8020.27 &  56 \\ 
 % Feature 21 & 8305.94 & 8364.04 & 8215.93 & 8273.93 & -8068.94 &  27 \\ 
 % Feature 22 & 8186.06 & 8243.98 & 8326.00 & 8383.94 & -8288.51 &   6 \\ 
 % Feature 23 & 8786.02 & 8844.10 & 8886.00 & 8943.99 & -8305.69 &  12 \\ 
 % Feature 24 & 8855.96 & 8913.97 & 8366.04 & 8424.04 & -8309.97 &   7 \\ 
 % Feature 25 & 8235.96 & 8294.04 & 8795.98 & 8853.95 & -8312.84 &  17 \\ 
 % Feature 26 & 8786.02 & 8844.10 & 8385.99 & 8443.94 & -8318.31 &  59 \\ 
 % Feature 27 & 8186.06 & 8243.98 & 8795.98 & 8853.95 & -8324.63 &   6 \\ 
 % Feature 28 & 8855.96 & 8913.97 & 8286.01 & 8343.92 & -8334.47 &  34 \\ 
   \hline
\end{tabular}
\caption {Recommended features and Continuum bandpass for predicting $ T_{eff} $ 
      by using BT\_Settl with SNR= $ 10 $ . 
      The Fitness and frequency of occurence are also included.} \label{tab:tab_SNR10_T} 
\end{center}
\end{table}


\begin{table}
\begin{center}
\begin{tabular}{rrrrrrr}
  \hline
 & Signal From & Signal To & Continuum From & Continuum To & Fitness & Freq \\ 
  \hline
Feature 1 & 8376.10 & 8433.91 & 9346.13 & 9403.92 & -6693.10 & 545 \\ 
  Feature 2 & 8286.01 & 8343.92 & 9186.03 & 9244.04 & -7250.83 &  17 \\ 
  Feature 3 & 8476.01 & 8534.03 & 9525.89 & 9584.05 & -7392.49 &  38 \\ 
  Feature 4 & 9276.00 & 9333.87 & 9425.95 & 9484.00 & -7578.17 &  19 \\ 
  Feature 5 & 9555.93 & 9614.06 & 8886.00 & 8943.99 & -7652.77 &  54 \\ 
 % Feature 6 & 9536.15 & 9593.96 & 8195.96 & 8254.03 & -7665.45 &  37 \\ 
 % Feature 7 & 8515.98 & 8573.99 & 8936.05 & 8994.03 & -7687.69 &  19 \\ 
 % Feature 8 & 8605.97 & 8663.96 & 8846.03 & 8904.03 & -7819.75 &  41 \\ 
 % Feature 9 & 9285.87 & 9344.05 & 9096.06 & 9154.07 & -7841.07 &  12 \\ 
 % Feature 10 & 8775.95 & 8833.94 & 9036.07 & 9094.04 & -7846.05 &  47 \\ 
 % Feature 11 & 9346.13 & 9403.92 & 8826.01 & 8883.94 & -7873.89 &  16 \\ 
 % Feature 12 & 9566.01 & 9623.96 & 9105.87 & 9163.91 & -7967.65 &   7 \\ 
 % Feature 13 & 9566.01 & 9623.96 & 9536.15 & 9593.96 & -7998.01 &  42 \\ 
 % Feature 14 & 8536.03 & 8594.06 & 8385.99 & 8443.94 & -8055.85 &  39 \\ 
 % Feature 15 & 8846.03 & 8904.03 & 8395.98 & 8453.99 & -8118.20 &  16 \\ 
 % Feature 16 & 9135.89 & 9193.92 & 9455.86 & 9514.14 & -8123.39 &  38 \\ 
 % Feature 17 & 8515.98 & 8573.99 & 8415.91 & 8473.96 & -8233.55 &  42 \\ 
 % Feature 18 & 8235.96 & 8294.04 & 8886.00 & 8943.99 & -8316.98 &   6 \\ 
 % Feature 19 & 8366.04 & 8424.04 & 8195.96 & 8254.03 & -8320.41 &  17 \\ 
 % Feature 20 & 8305.94 & 8364.04 & 8846.03 & 8904.03 & -8327.38 &  37 \\ 
 % Feature 21 & 8795.98 & 8853.95 & 8835.93 & 8893.97 & -8327.43 &  35 \\ 
 % Feature 22 & 8835.93 & 8893.97 & 8795.98 & 8853.95 & -8336.98 &  20 \\ 
   \hline
\end{tabular}
\caption {Recommended features and Continuum bandpass for predicting $ T_{eff} $ 
      by using BT\_Settl with SNR= $ 50 $ . 
      The Fitness and frequency of occurence are also included.} \label{tab:tab_SNR50_T} 
\end{center}
\end{table}


As in (\cite{2013A&A...549A.129C}) the authors provided their 
best estimation for suitable features, our interest is also to verify how
good it becomes in our particular case, as 
it can be an inderect assessment for the quality of the GA based recommendation. 

As a matter of reference the 
bandpass presented in Table ~\ref{tab:tab_cesetti} exploits the following bandpass:

\begin{table}
\begin{center}
\begin{tabular}{rrrrrrr}
  \hline
 & Signal\_from & Signal\_To & Cont1\_From & Cont1\_To & Cont2\_From & Cont2\_To \\ 
  \hline
  Pa1 & 8461 & 8474 & 8474 & 8484 & 8563 & 8577 \\ 
  Ca1 & 8484 & 8513 & 8474 & 8484 & 8563 & 8577 \\ 
  Ca2 & 8522 & 8562  & 8474 & 8484 & 8563 & 8577 \\ 
  Pa2 & 8577 & 8619 & 8563 & 8577 & 8619 & 8642 \\ 
  Ca3 & 8642 & 8682 & 8619 & 8642 & 8700 & 8725 \\ 
  Pa3 & 8730 & 8772 & 8700 & 8725 & 8776 & 8792 \\ 
  Mg & 8802 & 8811 & 8776 & 8792 & 8815 & 8850 \\ 
  Pa4 & 8850 & 8890 & 8815 & 8850 & 8890 & 8900 \\ 
  Pa5 & 9000 & 9030 & 8983 & 8998 & 9040 & 9050 \\
  FeClTi & 9080 & 9100 & 9040 & 9050 & 9125 & 9135 \\
  Pa6 & 9217 & 9255 & 9152 & 9165 & 9265 & 9275 \\
   \hline
\end{tabular}
\caption {Recommended features and Continuum bandpass recommended in 
   \cite{2013A&A...549A.129C} as relevant for temperature inside Band I}
   \label{tab:tab_cesetti} 
\end{center}
\end{table}


In regards with the Gravity, the GA recommends the features 
presentend in Table~\ref{tab:tab_SNRoo_G} for the pure synthetic signal.

\begin{table}
\begin{center}
\begin{tabular}{rrrrrrr}
  \hline
 & Signal From & Signal To & Continuum From & Continuum To & Fitness & Freq \\ 
  \hline
Feature 1 & 8176.03 & 8234.13 & 8295.99 & 8353.99 & -1244.00 & 278 \\ 
  Feature 2 & 8176.03 & 8234.13 & 8955.88 & 9013.95 & -1506.62 &   8 \\ 
  Feature 3 & 8636.06 & 8694.06 & 8536.03 & 8594.06 & -1515.42 &  14 \\ 
  Feature 4 & 8486.02 & 8544.05 & 8985.93 & 9043.98 & -1519.48 &  50 \\ 
  Feature 5 & 8496.05 & 8554.06 & 9436.02 & 9493.86 & -1520.16 &   6 \\ 
 % Feature 6 & 9085.96 & 9144.03 & 9276.00 & 9333.87 & -1522.24 &  15 \\ 
 % Feature 7 & 9315.88 & 9374.02 & 9566.01 & 9623.96 & -1524.35 &  11 \\ 
 % Feature 8 & 8985.93 & 9043.98 & 9285.87 & 9344.05 & -1527.06 &   9 \\ 
 % Feature 9 & 8245.98 & 8304.08 & 9006.05 & 9064.02 & -1529.28 &  29 \\ 
 % Feature 10 & 9156.03 & 9214.07 & 8376.10 & 8433.91 & -1532.13 &  17 \\ 
 % Feature 11 & 8215.93 & 8273.93 & 8596.11 & 8654.10 & -1534.68 &  21 \\ 
 % Feature 12 & 9276.00 & 9333.87 & 9395.94 & 9454.03 & -1537.74 &   6 \\ 
 % Feature 13 & 8235.96 & 8294.04 & 8205.98 & 8263.96 & -1540.04 &  33 \\ 
 % Feature 14 & 9576.03 & 9634.04 & 8145.92 & 8204.03 & -1541.89 &  61 \\ 
   \hline
\end{tabular}
\caption {Recommended features and Continuum bandpass for predicting $log(g)$ 
      by using BT\_Settl with SNR= $ \infty $ . 
      The Fitness and frequency of occurence are also included.} \label{tab:tab_SNRoo_G} 
\end{center}
\end{table}

The authors have produced the estimations for different SNR again 
as depicted in the tables \ref{tab:tab_SNR10_G} and \ref{tab:tab_SNR50_G}.

\begin{table}
\begin{center}
\begin{tabular}{rrrrrrr}
  \hline
 & Signal From & Signal To & Continuum From & Continuum To & Fitness & Freq \\ 
  \hline
Feature 1 & 8176.03 & 8234.13 & 8295.99 & 8353.99 & -1244.00 & 248 \\ 
  Feature 2 & 8176.03 & 8234.13 & 8305.94 & 8364.04 & -1252.85 &  16 \\ 
  Feature 3 & 8176.03 & 8234.13 & 8266.11 & 8324.03 & -1264.86 &   9 \\ 
  Feature 4 & 8556.06 & 8614.04 & 8716.00 & 8773.99 & -1489.47 &  33 \\ 
  Feature 5 & 8536.03 & 8594.06 & 9096.06 & 9154.07 & -1517.47 &  18 \\ 
 % Feature 6 & 9135.89 & 9193.92 & 9536.15 & 9593.96 & -1517.92 &  38 \\ 
 % Feature 7 & 8446.03 & 8503.94 & 9135.89 & 9193.92 & -1524.31 &   7 \\ 
 % Feature 8 & 8446.03 & 8503.94 & 9306.03 & 9363.93 & -1525.64 &   8 \\ 
 % Feature 9 & 9506.13 & 9563.85 & 9306.03 & 9363.93 & -1530.90 &  31 \\ 
 % Feature 10 & 8925.98 & 8983.96 & 9485.83 & 9544.11 & -1534.00 &   7 \\ 
 % Feature 11 & 9455.86 & 9514.14 & 9265.98 & 9323.99 & -1534.03 &  13 \\ 
 % Feature 12 & 8355.96 & 8414.03 & 8406.00 & 8464.07 & -1534.04 &   8 \\ 
 % Feature 13 & 9216.01 & 9274.05 & 8336.02 & 8394.08 & -1534.62 &   9 \\ 
 % Feature 14 & 8925.98 & 8983.96 & 9436.02 & 9493.86 & -1535.91 &  27 \\ 
 % Feature 15 & 9295.98 & 9354.08 & 8596.11 & 8654.10 & -1537.61 &  37 \\ 
 % Feature 16 & 9255.86 & 9314.01 & 9365.95 & 9424.02 & -1538.41 &   8 \\ 
 % Feature 17 & 8955.88 & 9013.95 & 9265.98 & 9323.99 & -1540.02 &  36 \\ 
 % Feature 18 & 8566.08 & 8624.07 & 8585.96 & 8643.95 & -1540.09 &  49 \\ 
 % Feature 19 & 9576.03 & 9634.04 & 9085.96 & 9144.03 & -1540.34 &  13 \\ 
 % Feature 20 & 8446.03 & 8503.94 & 8665.99 & 8723.96 & -1541.80 &   7 \\ 
 % Feature 21 & 8446.03 & 8503.94 & 8556.06 & 8614.04 & -1542.23 &  38 \\ 
 % Feature 22 & 8726.06 & 8784.07 & 8315.97 & 8374.00 & -1542.31 &  20 \\ 
   \hline
\end{tabular}
\caption {Recommended features and Continuum bandpass for predicting $log(g)$ 
      by using BT\_Settl with SNR= $ 10 $ . 
      The Fitness and frequency of occurence are also included.} \label{tab:tab_SNR10_G} 
\end{center}
\end{table}

\begin{table}
\begin{center}
\begin{tabular}{rrrrrrr}
  \hline
 & Signal From & Signal To & Continuum From & Continuum To & Fitness & Freq \\ 
  \hline
Feature 1 & 8176.03 & 8234.13 & 8205.98 & 8263.96 & -1320.54 &  50 \\ 
  Feature 2 & 8415.91 & 8473.96 & 8166.02 & 8224.12 & -1400.77 &  10 \\ 
  Feature 3 & 8645.93 & 8703.94 & 8665.99 & 8723.96 & -1422.86 &   8 \\ 
  Feature 4 & 8515.98 & 8573.99 & 8205.98 & 8263.96 & -1504.27 &   9 \\ 
  Feature 5 & 9425.95 & 9484.00 & 9146.00 & 9204.05 & -1512.67 &  13 \\ 
 % Feature 6 & 8486.02 & 8544.05 & 8936.05 & 8994.03 & -1517.66 &  10 \\ 
 % Feature 7 & 8476.01 & 8534.03 & 8976.09 & 9034.00 & -1522.11 &   7 \\ 
 % Feature 8 & 8446.03 & 8503.94 & 9455.86 & 9514.14 & -1526.30 &  15 \\ 
 % Feature 9 & 9156.03 & 9214.07 & 8346.02 & 8404.06 & -1529.63 &   9 \\ 
 % Feature 10 & 9636.16 & 9693.91 & 8215.93 & 8273.93 & -1531.01 &   9 \\ 
 % Feature 11 & 9135.89 & 9193.92 & 8385.99 & 8443.94 & -1531.25 &   9 \\ 
 % Feature 12 & 8865.98 & 8923.94 & 9536.15 & 9593.96 & -1532.35 &   8 \\ 
 % Feature 13 & 8665.99 & 8723.96 & 8685.95 & 8744.09 & -1532.53 &  21 \\ 
 % Feature 14 & 9126.00 & 9184.09 & 8215.93 & 8273.93 & -1533.15 &  10 \\ 
 % Feature 15 & 9525.89 & 9584.05 & 8355.96 & 8414.03 & -1534.59 &  17 \\ 
 % Feature 16 & 8616.00 & 8673.98 & 9406.09 & 9463.96 & -1534.81 &   9 \\ 
 % Feature 17 & 8626.02 & 8683.99 & 9335.79 & 9393.93 & -1534.98 &   8 \\ 
 % Feature 18 & 8305.94 & 8364.04 & 8286.01 & 8343.92 & -1535.35 &  13 \\ 
 % Feature 19 & 8685.95 & 8744.09 & 9096.06 & 9154.07 & -1535.57 &  19 \\ 
 % Feature 20 & 8636.06 & 8694.06 & 8235.96 & 8294.04 & -1536.95 &  12 \\ 
 % Feature 21 & 9395.94 & 9454.03 & 8826.01 & 8883.94 & -1539.98 &   6 \\ 
 % Feature 22 & 8195.96 & 8254.03 & 9115.99 & 9174.04 & -1540.19 &  15 \\ 
 % Feature 23 & 8786.02 & 8844.10 & 9235.98 & 9294.01 & -1542.00 &  10 \\  
   \hline
\end{tabular}
\caption {Recommended features and Continuum bandpass for predicting $log(g)$ 
      by using BT\_Settl with SNR= $ 50 $ . 
      The Fitness and frequency of occurence are also included.} \label{tab:tab_SNR50_G} 
\end{center}
\end{table}

Finally, features suggested for metallicity 
can be found in Table~\ref{tab:tab_SNRoo_M}.

\begin{table}
\begin{center}
\begin{tabular}{rrrrrrr}
  \hline
 & Signal From & Signal To & Continuum From & Continuum To & Fitness & Freq \\ 
  \hline
Feature 1 & 9085.96 & 9144.03 & 9445.97 & 9504.01  & -1146.72 & 348 \\ 
  Feature 2 & 9445.97 & 9504.01 & 9085.96 & 9144.03 & -1150.63 &  24 \\ 
  Feature 3 & 8556.06 & 8614.04 & 9135.89 & 9193.92 & -1209.47 &  11 \\ 
  Feature 4 & 9096.06 & 9154.07 & 8466.08 & 8523.98 & -1271.45 &   8 \\ 
  Feature 5 & 9045.91 & 9103.99 & 8525.91 & 8583.93 & -1276.04 &   5 \\  
   \hline
\end{tabular}
\caption {Recommended features and Continuum bandpass for predicting $Metallicity$ 
     by using BT\_Settl with SNR= $ \infty $ . 
      The Fitness and frequency of occurence are also included.} \label{tab:tab_SNRoo_M} 
\end{center}
\end{table}

And when different SNR are considered, the suggested features can be found in 
tables \ref{tab:tab_SNR10_M} and \ref{tab:tab_SNR50_M}

\begin{table}
\begin{center}
\begin{tabular}{rrrrrrr}
  \hline
 & Signal From & Signal To & Continuum From & Continuum To & Fitness & Freq \\ 
  \hline
Feature 1 & 9476.15 & 9534.00 & 9576.03 & 9634.04 & -1199.37 &  17 \\ 
  Feature 2 & 8466.08 & 8523.98 & 9146.00 & 9204.05 & -1207.84 & 189 \\ 
  Feature 3 & 8466.08 & 8523.98 & 9156.03 & 9214.07 & -1208.98 &  10 \\ 
  Feature 4 & 9466.08 & 9523.82 & 9416.04 & 9474.02 & -1220.23 &  12 \\ 
  Feature 5 & 9335.79 & 9393.93 & 9186.03 & 9244.04 & -1225.37 &  26 \\ 
 % Feature 6 & 8466.08 & 8523.98 & 8936.05 & 8994.03 & -1230.60 &   8 \\ 
 % Feature 7 & 9255.86 & 9314.01 & 9545.87 & 9604.02 & -1239.94 &  39 \\ 
 % Feature 8 & 9285.87 & 9344.05 & 9495.98 & 9553.95 & -1249.38 &   7 \\ 
 % Feature 9 & 9576.03 & 9634.04 & 8855.96 & 8913.97 & -1254.27 &   6 \\ 
 % Feature 10 & 9285.87 & 9344.05 & 8505.89 & 8563.93 & -1255.30 &   7 \\ 
 % Feature 11 & 9285.87 & 9344.05 & 9455.86 & 9514.14 & -1256.35 &  84 \\ 
 % Feature 12 & 9636.16 & 9693.91 & 8245.98 & 8304.08 & -1258.18 &   6 \\ 
 % Feature 13 & 9525.89 & 9584.05 & 9285.87 & 9344.05 & -1267.30 &  27 \\ 
 % Feature 14 & 9545.87 & 9604.02 & 8826.01 & 8883.94 & -1269.76 &  34 \\ 
 % Feature 15 & 9096.06 & 9154.07 & 8255.97 & 8314.06 & -1270.05 &   6 \\ 
 % Feature 16 & 8286.01 & 8343.92 & 9235.98 & 9294.01 & -1270.11 &   6 \\ 
 % Feature 17 & 8685.95 & 8744.09 & 8286.01 & 8343.92 & -1270.29 &  16 \\ 
 % Feature 18 & 8775.95 & 8833.94 & 8556.06 & 8614.04 & -1271.08 &   6 \\ 
 % Feature 19 & 9096.06 & 9154.07 & 9436.02 & 9493.86 & -1272.52 &  40 \\ 
 % Feature 20 & 9485.83 & 9544.11 & 8336.02 & 8394.08 & -1272.86 &  31 \\ 
 % Feature 21 & 8406.00 & 8464.07 & 9306.03 & 9363.93 & -1280.13 &  46 \\ 
 % Feature 22 & 9045.91 & 9103.99 & 8665.99 & 8723.96 & -1294.98 &   6 \\ 
 % Feature 23 & 8305.94 & 8364.04 & 8835.93 & 8893.97 & -1347.20 & 113 \\ 
 % Feature 24 & 8385.99 & 8443.94 & 8205.98 & 8263.96 & -1348.16 &   9 \\ 
 % Feature 25 & 8765.97 & 8823.95 & 8395.98 & 8453.99 & -1350.18 &  17 \\ 
   \hline
\end{tabular}
\caption {Recommended features and Continuum bandpass for predicting $Metallicity$ 
     by using BT\_Settl with SNR= $ 10 $ . 
      The Fitness and frequency of occurence are also included.} \label{tab:tab_SNR10_M} 
\end{center}
\end{table}

\begin{table}
\begin{center}
\begin{tabular}{rrrrrrr}
  \hline
 & Signal From & Signal To & Continuum From & Continuum To & Fitness & Freq \\ 
  \hline
Feature 1 & 9085.96 & 9144.03 & 9445.97 & 9504.01  & -1146.72 & 177 \\ 
  Feature 2 & 9445.97 & 9504.01 & 9085.96 & 9144.03 & -1150.63 &  5 \\ 
  Feature 3 & 8556.06 & 8614.04 & 9135.89 & 9193.92 & -1209.47 &  6 \\ 
  Feature 4 & 9096.06 & 9154.07 & 8466.08 & 8523.98 & -1271.45 &   6 \\ 
  Feature 5 & 9045.91 & 9103.99 & 8525.91 & 8583.93 & -1276.04 &   5 \\  
   \hline
\end{tabular}
\caption {Recommended features and Continuum bandpass for predicting $Metallicity$ 
     by using BT\_Settl with SNR= $ 50 $ . 
      The Fitness and frequency of occurence are also included.} \label{tab:tab_SNR50_M} 
\end{center}
\end{table}
  
  
}
