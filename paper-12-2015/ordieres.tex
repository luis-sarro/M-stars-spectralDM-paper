%                                                                 aa.dem
% AA vers. 8.2, LaTeX class for Astronomy & Astrophysics
% demonstration file
%                                                       (c) EDP Sciences
%-----------------------------------------------------------------------
%
%\documentclass[referee]{aa} % for a referee version
%\documentclass[onecolumn]{aa} % for a paper on 1 column  
%\documentclass[longauth]{aa} % for the long lists of affiliations 
%\documentclass[rnote]{aa} % for the research notes
%\documentclass[letter]{aa} % for the letters 
%\documentclass[bibyear]{aa} % if the references are not structured 
% according to the author-year natbib style

%
\documentclass[referee]{aa}  

%
\usepackage{graphicx}
\usepackage{caption}
\usepackage{subcaption}
%%%%%%%%%%%%%%%%%%%%%%%%%%%%%%%%%%%%%%%%
\usepackage{txfonts}
%%%%%%%%%%%%%%%%%%%%%%%%%%%%%%%%%%%%%%%%
%\usepackage[options]{hyperref}
% To add links in your PDF file, use the package "hyperref"
% with options according to your LaTeX or PDFLaTeX drivers.
%
% Enable mutiple title row on tables.
\usepackage{multirow}
%%%%%%%%%%%
% 
% Dealing with ``too many unprocessed floats'' use it with \clearpage 
\usepackage[section] {placeins}

% For marginal notes
\usepackage{marginnote}

\begin{document} 


   \title{Physical parameter estimates of M-type stars: a machine learning perspective.}

   \author{
          J. Ordieres-Mere \inst{2} 
          \and
          A. Bello-Garcia\inst{3} 
          \and
          A. Gonzalez-Marcos\inst{4} 
          \and
          M.B. Prendes-Gero\inst{3} 
          \and
          L. M. Sarro \inst{1}  
          }

   \institute{ 
      \inst{1} Universidad Nacional de Educaci\'{o}n a Distancia, \\
               Department of Artificial Intelligence.
               \email{lsb@uned.es}
        \and 
      \inst{2}  Universidad Polit\'{e}cnica de Madrid (UPM), PMQ Research Group, \\
              Jos\'{e} Guti\'{e}rrez Abascal 2, 28006 Madrid, Spain. 
              \email{j.ordieres@upm.es}
        \and 
      \inst{3}   Universidad de Oviedo, Construction and Manufacturing Engineering Department, \\
              Campus de Viesques s/n, Gij\'{o}n, Asturias, Spain. 
              \email{\{abello,mbprendes\}@uniovi.es} 
        \and 
      \inst{4}  Universidad de la Rioja, P2ML Research Group, \\
              Luis de Ulloa 20, 26004 Logro\~{n}o, La Rioja, Spain. 
              \email{ana.gonzalez@unirioja.es}
      }

   \date{Received \today; accepted }

% \abstract{}{}{}{}{} 
% 5 {} token are mandatory
 
  \abstract
  % context heading (optional)
  % {} leave it empty if necessary  
%    {To investigate the physical nature of the `nuc\-leated instability' of
%    proto giant planets, the stability of layers
%    in static, radiative gas spheres is analysed on the basis of Baker's
%    standard one-zone model.}
%   % aims heading (mandatory)
%    {It is shown that stability
%    depends only upon the equations of state, the opacities and the local
%    thermodynamic state in the layer. Stability and instability can
%    therefore be expressed in the form of stability equations of state
%    which are universal for a given composition.}
%   % methods heading (mandatory)
%    {The stability equations of state are
%    calculated for solar composition and are displayed in the domain
%    $-14 \leq \lg \rho / \mathrm{[g\, cm^{-3}]} \leq 0 $,
%    $ 8.8 \leq \lg e / \mathrm{[erg\, g^{-1}]} \leq 17.7$. These displays
%    may be
%    used to determine the one-zone stability of layers in stellar
%    or planetary structure models by directly reading off the value of
%    the stability equations for the thermodynamic state of these layers,
%    specified
%    by state quantities as density $\rho$, temperature $T$ or
%    specific internal energy $e$.
%    Regions of instability in the $(\rho,e)$-plane are described
%    and related to the underlying microphysical processes.}
%   % results heading (mandatory)
%    {Vibrational instability is found to be a common phenomenon
%    at temperatures lower than the second He ionisation
%    zone. The $\kappa$-mechanism is widespread under `cool'
%    conditions.}
%   % conclusions heading (optional), leave it empty if necessary 
   {}

   \keywords{class M stars --
                dynamic feature selection --
                physical parameter identification --
                Temperature, gravity and metalicity Modelling --
                Learning from BT-Settl spectra library
               }

   \maketitle
%

\section{Introduction}
\label{sec:intro}

% The importance of M stars
% The problem of estimation of stellar parameters in the M regime: bands, no continuum...
% who and what. Teff calibrations
% \cite{rajpurohit}
% \cite{amelia}

% Atlases 
%\cite{2013arXiv1306.3709B} % for lambda > 1.1 um :(
%\cite{2009ApJS..185..289R} %IRTF library of cool star

%\cite{2012ApJ...748...93R} % Metall. and Teff indicators in the K band! :(

% Summary of spectral diagnostics in the IR for M stars


\section{Methodology. \label{meth}}
\input{meth}

\section{Physical parameters of the IRTF collection of spectra.}
%\input{irtf}
\subsection{Spectral bands selected}

During the preprocessing stage (described in Sect. \ref{meth}) the
spectral resolution of the BT-Settl library was degraded to the IRTF
resolution (R ~ 2000) by convolving with a Gaussian. Then, the spectra
were trimmed to produce valid segments between 8145.92 and
24106.85{\AA}, which is the spectral range common to all M stars in
the IRTF library. Finally, all spectra were divided by the total
integrated flux in this range in order to factor out the stellar
distance.}

The application of the GAs to the selection of features for the
prediction of effective temperature from noiseless spectra with the
IRTF wavelength range and resolution results in the features included
in Table~\ref{tab:tab_NC_T}. Features are ordered by the fitness value
(the AIC) and we only consider features that are present in at least 5
sets.

\begin{table}
\begin{center}
\begin{tabular}{rrrrrrr}
  \hline
 & $\lambda_1$ & $\lambda_2$ & $\lambda_3$ & $\lambda_4 $ & Fitness (AIC) & Freq. \\ 
  \hline
Feature 1 & 8376.10 & 8433.91 & 9346.13 & 9403.92 & -6693.10 & 319 \\ 
Feature 2 & 8385.99 & 8443.94 & 9346.13 & 9403.92 & -6700.15 &   6 \\ 
Feature 3 & 8195.96 & 8254.03 & 9386.01 & 9444.05 & -6887.55 &  44 \\ 
Feature 4 & 8186.06 & 8243.98 & 9235.98 & 9294.01 & -7056.23 &  19 \\ 
Feature 5 & 8406.00 & 8464.07 & 9515.96 & 9574.13 & -7068.86 &  34 \\ 
%  Feature 6 & 9326.07 & 9384.15 & 8406.00 & 8464.07 & -7349.99 &  18 \\ 
%  Feature 7 & 8496.05 & 8554.06 & 9576.03 & 9634.04 & -7505.21 &  38 \\ 
%  Feature 8 & 9036.07 & 9094.04 & 9075.93 & 9133.98 & -7535.71 &  15 \\ 
%  Feature 9 & 9135.89 & 9193.92 & 9085.96 & 9144.03 & -7622.16 &  27 \\ 
%  Feature 10 & 9515.96 & 9574.13 & 8876.08 & 8934.03 & -7655.59 &   6 \\ 
%  Feature 11 & 8716.00 & 8773.99 & 9025.93 & 9084.07 & -7703.40 &  77 \\ 
%  Feature 12 & 9156.03 & 9214.07 & 8255.97 & 8314.06 & -7708.62 &  15 \\ 
%  Feature 13 & 8266.11 & 8324.03 & 8235.96 & 8294.04 & -7856.20 &  30 \\ 
%  Feature 14 & 8235.96 & 8294.04 & 8255.97 & 8314.06 & -7860.73 &  69 \\ 
%  Feature 15 & 8705.93 & 8763.97 & 8886.00 & 8943.99 & -7919.62 &  27 \\ 
%  Feature 16 & 8536.03 & 8594.06 & 8336.02 & 8394.08 & -7940.67 &  24 \\ 
%  Feature 17 & 8605.97 & 8663.96 & 8346.02 & 8404.06 & -7953.63 &  59 \\ 
%  Feature 18 & 8946.07 & 9004.01 & 8756.09 & 8814.06 & -8117.90 &  46 \\ 
%  Feature 19 & 9135.89 & 9193.92 & 9485.83 & 9544.11 & -8211.98 &  36 \\ 
%  Feature 20 & 8536.03 & 8594.06 & 8496.05 & 8554.06 & -8337.36 &  31 \\ 
   \hline
\end{tabular}
\caption {Features selected by the GA for predicting $T_{eff}$ 
      using BT\_Settl noiseless synthetic
      spectra. } \label{tab:tab_NC_T}
\end{center}
\end{table}

{\bf TBD by Luis: interpret the features.}

When noise is added to the BT-Settl spectra, we obtain 

%The authors have estimated the suggested features when theoretical BT\_Settl 
%is noised with different SNR and following tables \ref{tab:tab_SNR10_T} 
%and \ref{tab:tab_SNR50_T} sumarize the findings.

\begin{table}
\begin{center}
\begin{tabular}{rrrrrrr}
  \hline
 & Signal From & Signal To & Continuum From & Continuum To & Fitness & Freq \\ 
  \hline
Feature 1 & 8385.99 & 8443.94 & 9395.94 & 9454.03 & -6734.59 & 136 \\ 
Feature 2 & 8186.06 & 8243.98 & 9536.15 & 9593.96 & -6857.65 &   7 \\ 
Feature 3 & 8186.06 & 8243.98 & 9376.07 & 9433.92 & -6947.54 &   7 \\ 
Feature 4 & 8286.01 & 8343.92 & 9206.05 & 9264.00 & -7123.10 &  10 \\ 
Feature 5 & 8355.96 & 8414.03 & 9066.05 & 9124.05 & -7207.55 &  37 \\ 
 % Feature 6 & 9276.00 & 9333.87 & 8415.91 & 8473.96 & -7436.13 &  32 \\ 
 % Feature 7 & 8455.96 & 8513.93 & 9055.94 & 9114.07 & -7476.12 &  32 \\ 
 % Feature 8 & 8616.00 & 8673.98 & 9576.03 & 9634.04 & -7491.24 &   6 \\ 
 % Feature 9 & 8536.03 & 8594.06 & 9135.89 & 9193.92 & -7573.72 &  58 \\ 
 % Feature 10 & 9395.94 & 9454.03 & 9356.05 & 9414.08 & -7586.78 &  45 \\ 
 % Feature 11 & 9576.03 & 9634.04 & 9045.91 & 9103.99 & -7641.39 &  19 \\ 
 % Feature 12 & 9066.05 & 9124.05 & 8166.02 & 8224.12 & -7642.28 &  15 \\ 
 % Feature 13 & 9386.01 & 9444.05 & 9536.15 & 9593.96 & -7684.66 &   7 \\ 
 % Feature 14 & 8616.00 & 8673.98 & 8756.09 & 8814.06 & -7686.46 &  17 \\ 
 % Feature 15 & 9576.03 & 9634.04 & 8145.92 & 8204.03 & -7767.19 &  13 \\ 
 % Feature 16 & 9466.08 & 9523.82 & 9036.07 & 9094.04 & -7772.45 &  37 \\ 
 % Feature 17 & 9445.97 & 9504.01 & 9545.87 & 9604.02 & -7830.60 &  35 \\ 
 % Feature 18 & 8286.01 & 8343.92 & 8486.02 & 8544.05 & -7863.15 &  49 \\ 
 % Feature 19 & 9186.03 & 9244.04 & 9135.89 & 9193.92 & -7884.30 &  13 \\ 
 % Feature 20 & 9306.03 & 9363.93 & 8745.93 & 8803.93 & -8020.27 &  56 \\ 
 % Feature 21 & 8305.94 & 8364.04 & 8215.93 & 8273.93 & -8068.94 &  27 \\ 
 % Feature 22 & 8186.06 & 8243.98 & 8326.00 & 8383.94 & -8288.51 &   6 \\ 
 % Feature 23 & 8786.02 & 8844.10 & 8886.00 & 8943.99 & -8305.69 &  12 \\ 
 % Feature 24 & 8855.96 & 8913.97 & 8366.04 & 8424.04 & -8309.97 &   7 \\ 
 % Feature 25 & 8235.96 & 8294.04 & 8795.98 & 8853.95 & -8312.84 &  17 \\ 
 % Feature 26 & 8786.02 & 8844.10 & 8385.99 & 8443.94 & -8318.31 &  59 \\ 
 % Feature 27 & 8186.06 & 8243.98 & 8795.98 & 8853.95 & -8324.63 &   6 \\ 
 % Feature 28 & 8855.96 & 8913.97 & 8286.01 & 8343.92 & -8334.47 &  34 \\ 
   \hline
\end{tabular}
\caption {Recommended features and Continuum bandpass for predicting $ T_{eff} $ 
      by using BT\_Settl with SNR= $ 10 $ . 
      The Fitness and frequency of occurence are also included.} \label{tab:tab_SNR10_T} 
\end{center}
\end{table}


\begin{table}
\begin{center}
\begin{tabular}{rrrrrrr}
  \hline
 & Signal From & Signal To & Continuum From & Continuum To & Fitness & Freq \\ 
  \hline
Feature 1 & 8376.10 & 8433.91 & 9346.13 & 9403.92 & -6693.10 & 545 \\ 
  Feature 2 & 8286.01 & 8343.92 & 9186.03 & 9244.04 & -7250.83 &  17 \\ 
  Feature 3 & 8476.01 & 8534.03 & 9525.89 & 9584.05 & -7392.49 &  38 \\ 
  Feature 4 & 9276.00 & 9333.87 & 9425.95 & 9484.00 & -7578.17 &  19 \\ 
  Feature 5 & 9555.93 & 9614.06 & 8886.00 & 8943.99 & -7652.77 &  54 \\ 
 % Feature 6 & 9536.15 & 9593.96 & 8195.96 & 8254.03 & -7665.45 &  37 \\ 
 % Feature 7 & 8515.98 & 8573.99 & 8936.05 & 8994.03 & -7687.69 &  19 \\ 
 % Feature 8 & 8605.97 & 8663.96 & 8846.03 & 8904.03 & -7819.75 &  41 \\ 
 % Feature 9 & 9285.87 & 9344.05 & 9096.06 & 9154.07 & -7841.07 &  12 \\ 
 % Feature 10 & 8775.95 & 8833.94 & 9036.07 & 9094.04 & -7846.05 &  47 \\ 
 % Feature 11 & 9346.13 & 9403.92 & 8826.01 & 8883.94 & -7873.89 &  16 \\ 
 % Feature 12 & 9566.01 & 9623.96 & 9105.87 & 9163.91 & -7967.65 &   7 \\ 
 % Feature 13 & 9566.01 & 9623.96 & 9536.15 & 9593.96 & -7998.01 &  42 \\ 
 % Feature 14 & 8536.03 & 8594.06 & 8385.99 & 8443.94 & -8055.85 &  39 \\ 
 % Feature 15 & 8846.03 & 8904.03 & 8395.98 & 8453.99 & -8118.20 &  16 \\ 
 % Feature 16 & 9135.89 & 9193.92 & 9455.86 & 9514.14 & -8123.39 &  38 \\ 
 % Feature 17 & 8515.98 & 8573.99 & 8415.91 & 8473.96 & -8233.55 &  42 \\ 
 % Feature 18 & 8235.96 & 8294.04 & 8886.00 & 8943.99 & -8316.98 &   6 \\ 
 % Feature 19 & 8366.04 & 8424.04 & 8195.96 & 8254.03 & -8320.41 &  17 \\ 
 % Feature 20 & 8305.94 & 8364.04 & 8846.03 & 8904.03 & -8327.38 &  37 \\ 
 % Feature 21 & 8795.98 & 8853.95 & 8835.93 & 8893.97 & -8327.43 &  35 \\ 
 % Feature 22 & 8835.93 & 8893.97 & 8795.98 & 8853.95 & -8336.98 &  20 \\ 
   \hline
\end{tabular}
\caption {Recommended features and Continuum bandpass for predicting $ T_{eff} $ 
      by using BT\_Settl with SNR= $ 50 $ . 
      The Fitness and frequency of occurence are also included.} \label{tab:tab_SNR50_T} 
\end{center}
\end{table}


As in (\cite{2013A&A...549A.129C}) the authors provided their 
best estimation for suitable features, our interest is also to verify how
good it becomes in our particular case, as 
it can be an inderect assessment for the quality of the GA based recommendation. 

As a matter of reference the 
bandpass presented in Table ~\ref{tab:tab_cesetti} exploits the following bandpass:

\begin{table}
\begin{center}
\begin{tabular}{rrrrrrr}
  \hline
 & Signal\_from & Signal\_To & Cont1\_From & Cont1\_To & Cont2\_From & Cont2\_To \\ 
  \hline
  Pa1 & 8461 & 8474 & 8474 & 8484 & 8563 & 8577 \\ 
  Ca1 & 8484 & 8513 & 8474 & 8484 & 8563 & 8577 \\ 
  Ca2 & 8522 & 8562  & 8474 & 8484 & 8563 & 8577 \\ 
  Pa2 & 8577 & 8619 & 8563 & 8577 & 8619 & 8642 \\ 
  Ca3 & 8642 & 8682 & 8619 & 8642 & 8700 & 8725 \\ 
  Pa3 & 8730 & 8772 & 8700 & 8725 & 8776 & 8792 \\ 
  Mg & 8802 & 8811 & 8776 & 8792 & 8815 & 8850 \\ 
  Pa4 & 8850 & 8890 & 8815 & 8850 & 8890 & 8900 \\ 
  Pa5 & 9000 & 9030 & 8983 & 8998 & 9040 & 9050 \\
  FeClTi & 9080 & 9100 & 9040 & 9050 & 9125 & 9135 \\
  Pa6 & 9217 & 9255 & 9152 & 9165 & 9265 & 9275 \\
   \hline
\end{tabular}
\caption {Recommended features and Continuum bandpass recommended in 
   \cite{2013A&A...549A.129C} as relevant for temperature inside Band I}
   \label{tab:tab_cesetti} 
\end{center}
\end{table}


In regards with the Gravity, the GA recommends the features 
presentend in Table~\ref{tab:tab_SNRoo_G} for the pure synthetic signal.

\begin{table}
\begin{center}
\begin{tabular}{rrrrrrr}
  \hline
 & Signal From & Signal To & Continuum From & Continuum To & Fitness & Freq \\ 
  \hline
Feature 1 & 8176.03 & 8234.13 & 8295.99 & 8353.99 & -1244.00 & 278 \\ 
  Feature 2 & 8176.03 & 8234.13 & 8955.88 & 9013.95 & -1506.62 &   8 \\ 
  Feature 3 & 8636.06 & 8694.06 & 8536.03 & 8594.06 & -1515.42 &  14 \\ 
  Feature 4 & 8486.02 & 8544.05 & 8985.93 & 9043.98 & -1519.48 &  50 \\ 
  Feature 5 & 8496.05 & 8554.06 & 9436.02 & 9493.86 & -1520.16 &   6 \\ 
 % Feature 6 & 9085.96 & 9144.03 & 9276.00 & 9333.87 & -1522.24 &  15 \\ 
 % Feature 7 & 9315.88 & 9374.02 & 9566.01 & 9623.96 & -1524.35 &  11 \\ 
 % Feature 8 & 8985.93 & 9043.98 & 9285.87 & 9344.05 & -1527.06 &   9 \\ 
 % Feature 9 & 8245.98 & 8304.08 & 9006.05 & 9064.02 & -1529.28 &  29 \\ 
 % Feature 10 & 9156.03 & 9214.07 & 8376.10 & 8433.91 & -1532.13 &  17 \\ 
 % Feature 11 & 8215.93 & 8273.93 & 8596.11 & 8654.10 & -1534.68 &  21 \\ 
 % Feature 12 & 9276.00 & 9333.87 & 9395.94 & 9454.03 & -1537.74 &   6 \\ 
 % Feature 13 & 8235.96 & 8294.04 & 8205.98 & 8263.96 & -1540.04 &  33 \\ 
 % Feature 14 & 9576.03 & 9634.04 & 8145.92 & 8204.03 & -1541.89 &  61 \\ 
   \hline
\end{tabular}
\caption {Recommended features and Continuum bandpass for predicting $log(g)$ 
      by using BT\_Settl with SNR= $ \infty $ . 
      The Fitness and frequency of occurence are also included.} \label{tab:tab_SNRoo_G} 
\end{center}
\end{table}

The authors have produced the estimations for different SNR again 
as depicted in the tables \ref{tab:tab_SNR10_G} and \ref{tab:tab_SNR50_G}.

\begin{table}
\begin{center}
\begin{tabular}{rrrrrrr}
  \hline
 & Signal From & Signal To & Continuum From & Continuum To & Fitness & Freq \\ 
  \hline
Feature 1 & 8176.03 & 8234.13 & 8295.99 & 8353.99 & -1244.00 & 248 \\ 
  Feature 2 & 8176.03 & 8234.13 & 8305.94 & 8364.04 & -1252.85 &  16 \\ 
  Feature 3 & 8176.03 & 8234.13 & 8266.11 & 8324.03 & -1264.86 &   9 \\ 
  Feature 4 & 8556.06 & 8614.04 & 8716.00 & 8773.99 & -1489.47 &  33 \\ 
  Feature 5 & 8536.03 & 8594.06 & 9096.06 & 9154.07 & -1517.47 &  18 \\ 
 % Feature 6 & 9135.89 & 9193.92 & 9536.15 & 9593.96 & -1517.92 &  38 \\ 
 % Feature 7 & 8446.03 & 8503.94 & 9135.89 & 9193.92 & -1524.31 &   7 \\ 
 % Feature 8 & 8446.03 & 8503.94 & 9306.03 & 9363.93 & -1525.64 &   8 \\ 
 % Feature 9 & 9506.13 & 9563.85 & 9306.03 & 9363.93 & -1530.90 &  31 \\ 
 % Feature 10 & 8925.98 & 8983.96 & 9485.83 & 9544.11 & -1534.00 &   7 \\ 
 % Feature 11 & 9455.86 & 9514.14 & 9265.98 & 9323.99 & -1534.03 &  13 \\ 
 % Feature 12 & 8355.96 & 8414.03 & 8406.00 & 8464.07 & -1534.04 &   8 \\ 
 % Feature 13 & 9216.01 & 9274.05 & 8336.02 & 8394.08 & -1534.62 &   9 \\ 
 % Feature 14 & 8925.98 & 8983.96 & 9436.02 & 9493.86 & -1535.91 &  27 \\ 
 % Feature 15 & 9295.98 & 9354.08 & 8596.11 & 8654.10 & -1537.61 &  37 \\ 
 % Feature 16 & 9255.86 & 9314.01 & 9365.95 & 9424.02 & -1538.41 &   8 \\ 
 % Feature 17 & 8955.88 & 9013.95 & 9265.98 & 9323.99 & -1540.02 &  36 \\ 
 % Feature 18 & 8566.08 & 8624.07 & 8585.96 & 8643.95 & -1540.09 &  49 \\ 
 % Feature 19 & 9576.03 & 9634.04 & 9085.96 & 9144.03 & -1540.34 &  13 \\ 
 % Feature 20 & 8446.03 & 8503.94 & 8665.99 & 8723.96 & -1541.80 &   7 \\ 
 % Feature 21 & 8446.03 & 8503.94 & 8556.06 & 8614.04 & -1542.23 &  38 \\ 
 % Feature 22 & 8726.06 & 8784.07 & 8315.97 & 8374.00 & -1542.31 &  20 \\ 
   \hline
\end{tabular}
\caption {Recommended features and Continuum bandpass for predicting $log(g)$ 
      by using BT\_Settl with SNR= $ 10 $ . 
      The Fitness and frequency of occurence are also included.} \label{tab:tab_SNR10_G} 
\end{center}
\end{table}

\begin{table}
\begin{center}
\begin{tabular}{rrrrrrr}
  \hline
 & Signal From & Signal To & Continuum From & Continuum To & Fitness & Freq \\ 
  \hline
Feature 1 & 8176.03 & 8234.13 & 8205.98 & 8263.96 & -1320.54 &  50 \\ 
  Feature 2 & 8415.91 & 8473.96 & 8166.02 & 8224.12 & -1400.77 &  10 \\ 
  Feature 3 & 8645.93 & 8703.94 & 8665.99 & 8723.96 & -1422.86 &   8 \\ 
  Feature 4 & 8515.98 & 8573.99 & 8205.98 & 8263.96 & -1504.27 &   9 \\ 
  Feature 5 & 9425.95 & 9484.00 & 9146.00 & 9204.05 & -1512.67 &  13 \\ 
 % Feature 6 & 8486.02 & 8544.05 & 8936.05 & 8994.03 & -1517.66 &  10 \\ 
 % Feature 7 & 8476.01 & 8534.03 & 8976.09 & 9034.00 & -1522.11 &   7 \\ 
 % Feature 8 & 8446.03 & 8503.94 & 9455.86 & 9514.14 & -1526.30 &  15 \\ 
 % Feature 9 & 9156.03 & 9214.07 & 8346.02 & 8404.06 & -1529.63 &   9 \\ 
 % Feature 10 & 9636.16 & 9693.91 & 8215.93 & 8273.93 & -1531.01 &   9 \\ 
 % Feature 11 & 9135.89 & 9193.92 & 8385.99 & 8443.94 & -1531.25 &   9 \\ 
 % Feature 12 & 8865.98 & 8923.94 & 9536.15 & 9593.96 & -1532.35 &   8 \\ 
 % Feature 13 & 8665.99 & 8723.96 & 8685.95 & 8744.09 & -1532.53 &  21 \\ 
 % Feature 14 & 9126.00 & 9184.09 & 8215.93 & 8273.93 & -1533.15 &  10 \\ 
 % Feature 15 & 9525.89 & 9584.05 & 8355.96 & 8414.03 & -1534.59 &  17 \\ 
 % Feature 16 & 8616.00 & 8673.98 & 9406.09 & 9463.96 & -1534.81 &   9 \\ 
 % Feature 17 & 8626.02 & 8683.99 & 9335.79 & 9393.93 & -1534.98 &   8 \\ 
 % Feature 18 & 8305.94 & 8364.04 & 8286.01 & 8343.92 & -1535.35 &  13 \\ 
 % Feature 19 & 8685.95 & 8744.09 & 9096.06 & 9154.07 & -1535.57 &  19 \\ 
 % Feature 20 & 8636.06 & 8694.06 & 8235.96 & 8294.04 & -1536.95 &  12 \\ 
 % Feature 21 & 9395.94 & 9454.03 & 8826.01 & 8883.94 & -1539.98 &   6 \\ 
 % Feature 22 & 8195.96 & 8254.03 & 9115.99 & 9174.04 & -1540.19 &  15 \\ 
 % Feature 23 & 8786.02 & 8844.10 & 9235.98 & 9294.01 & -1542.00 &  10 \\  
   \hline
\end{tabular}
\caption {Recommended features and Continuum bandpass for predicting $log(g)$ 
      by using BT\_Settl with SNR= $ 50 $ . 
      The Fitness and frequency of occurence are also included.} \label{tab:tab_SNR50_G} 
\end{center}
\end{table}

Finally, features suggested for metallicity 
can be found in Table~\ref{tab:tab_SNRoo_M}.

\begin{table}
\begin{center}
\begin{tabular}{rrrrrrr}
  \hline
 & Signal From & Signal To & Continuum From & Continuum To & Fitness & Freq \\ 
  \hline
Feature 1 & 9085.96 & 9144.03 & 9445.97 & 9504.01  & -1146.72 & 348 \\ 
  Feature 2 & 9445.97 & 9504.01 & 9085.96 & 9144.03 & -1150.63 &  24 \\ 
  Feature 3 & 8556.06 & 8614.04 & 9135.89 & 9193.92 & -1209.47 &  11 \\ 
  Feature 4 & 9096.06 & 9154.07 & 8466.08 & 8523.98 & -1271.45 &   8 \\ 
  Feature 5 & 9045.91 & 9103.99 & 8525.91 & 8583.93 & -1276.04 &   5 \\  
   \hline
\end{tabular}
\caption {Recommended features and Continuum bandpass for predicting $Metallicity$ 
     by using BT\_Settl with SNR= $ \infty $ . 
      The Fitness and frequency of occurence are also included.} \label{tab:tab_SNRoo_M} 
\end{center}
\end{table}

And when different SNR are considered, the suggested features can be found in 
tables \ref{tab:tab_SNR10_M} and \ref{tab:tab_SNR50_M}

\begin{table}
\begin{center}
\begin{tabular}{rrrrrrr}
  \hline
 & Signal From & Signal To & Continuum From & Continuum To & Fitness & Freq \\ 
  \hline
Feature 1 & 9476.15 & 9534.00 & 9576.03 & 9634.04 & -1199.37 &  17 \\ 
  Feature 2 & 8466.08 & 8523.98 & 9146.00 & 9204.05 & -1207.84 & 189 \\ 
  Feature 3 & 8466.08 & 8523.98 & 9156.03 & 9214.07 & -1208.98 &  10 \\ 
  Feature 4 & 9466.08 & 9523.82 & 9416.04 & 9474.02 & -1220.23 &  12 \\ 
  Feature 5 & 9335.79 & 9393.93 & 9186.03 & 9244.04 & -1225.37 &  26 \\ 
 % Feature 6 & 8466.08 & 8523.98 & 8936.05 & 8994.03 & -1230.60 &   8 \\ 
 % Feature 7 & 9255.86 & 9314.01 & 9545.87 & 9604.02 & -1239.94 &  39 \\ 
 % Feature 8 & 9285.87 & 9344.05 & 9495.98 & 9553.95 & -1249.38 &   7 \\ 
 % Feature 9 & 9576.03 & 9634.04 & 8855.96 & 8913.97 & -1254.27 &   6 \\ 
 % Feature 10 & 9285.87 & 9344.05 & 8505.89 & 8563.93 & -1255.30 &   7 \\ 
 % Feature 11 & 9285.87 & 9344.05 & 9455.86 & 9514.14 & -1256.35 &  84 \\ 
 % Feature 12 & 9636.16 & 9693.91 & 8245.98 & 8304.08 & -1258.18 &   6 \\ 
 % Feature 13 & 9525.89 & 9584.05 & 9285.87 & 9344.05 & -1267.30 &  27 \\ 
 % Feature 14 & 9545.87 & 9604.02 & 8826.01 & 8883.94 & -1269.76 &  34 \\ 
 % Feature 15 & 9096.06 & 9154.07 & 8255.97 & 8314.06 & -1270.05 &   6 \\ 
 % Feature 16 & 8286.01 & 8343.92 & 9235.98 & 9294.01 & -1270.11 &   6 \\ 
 % Feature 17 & 8685.95 & 8744.09 & 8286.01 & 8343.92 & -1270.29 &  16 \\ 
 % Feature 18 & 8775.95 & 8833.94 & 8556.06 & 8614.04 & -1271.08 &   6 \\ 
 % Feature 19 & 9096.06 & 9154.07 & 9436.02 & 9493.86 & -1272.52 &  40 \\ 
 % Feature 20 & 9485.83 & 9544.11 & 8336.02 & 8394.08 & -1272.86 &  31 \\ 
 % Feature 21 & 8406.00 & 8464.07 & 9306.03 & 9363.93 & -1280.13 &  46 \\ 
 % Feature 22 & 9045.91 & 9103.99 & 8665.99 & 8723.96 & -1294.98 &   6 \\ 
 % Feature 23 & 8305.94 & 8364.04 & 8835.93 & 8893.97 & -1347.20 & 113 \\ 
 % Feature 24 & 8385.99 & 8443.94 & 8205.98 & 8263.96 & -1348.16 &   9 \\ 
 % Feature 25 & 8765.97 & 8823.95 & 8395.98 & 8453.99 & -1350.18 &  17 \\ 
   \hline
\end{tabular}
\caption {Recommended features and Continuum bandpass for predicting $Metallicity$ 
     by using BT\_Settl with SNR= $ 10 $ . 
      The Fitness and frequency of occurence are also included.} \label{tab:tab_SNR10_M} 
\end{center}
\end{table}

\begin{table}
\begin{center}
\begin{tabular}{rrrrrrr}
  \hline
 & Signal From & Signal To & Continuum From & Continuum To & Fitness & Freq \\ 
  \hline
Feature 1 & 9085.96 & 9144.03 & 9445.97 & 9504.01  & -1146.72 & 177 \\ 
  Feature 2 & 9445.97 & 9504.01 & 9085.96 & 9144.03 & -1150.63 &  5 \\ 
  Feature 3 & 8556.06 & 8614.04 & 9135.89 & 9193.92 & -1209.47 &  6 \\ 
  Feature 4 & 9096.06 & 9154.07 & 8466.08 & 8523.98 & -1271.45 &   6 \\ 
  Feature 5 & 9045.91 & 9103.99 & 8525.91 & 8583.93 & -1276.04 &   5 \\  
   \hline
\end{tabular}
\caption {Recommended features and Continuum bandpass for predicting $Metallicity$ 
     by using BT\_Settl with SNR= $ 50 $ . 
      The Fitness and frequency of occurence are also included.} \label{tab:tab_SNR50_M} 
\end{center}
\end{table}
  

\subsection{Regression models}

In the following, we will summarise the results obtained for the IRTF
data set. We deal with the different physical paramenters in separate
Sections. We start by reporting the Root Mean/Median Square Errors
(RMSE/RMDSE) with respect to the parameters gathered from the
literature by \cite{cesetti} and included in their Table 3.

\subsubsection{Effective temperature models}

Table \ref{tab:model_TSD} summarises the RMSE/RMDSE for the complete
set of models: the minimum $\chi^2$ estimate based on the full
spectrum ($\chi^2$), the projection pursuit regression based on the
ICA components (PPR-ICA) and models trained on the spectral features
proposed by the GA (GA-RF, GA-GBM, GA-SVR, GA-NNET, GA-MARS,
GA-KPLS, GA-RR). For each model, we report the RMSE/RMDSE obtained for
several noise levels of the training sets.  SNR=$\infty$ corresponds
to noiseless spectra. In the GA- cases, the model is trained with
  the spectral features found by the Genetic Algorithms when applied
  to BT-Settl spectra of the corresponding SNR.

{\bf Make sure we always have Rule-Regression models everywhere or
  discuss why not.}
  
\newcommand{\ra}[1]{\renewcommand{\arraystretch}{#1}}
\begin{table*}\centering
\ra{1.3}
\begin{tabular}{@{}lrrcrrcrr@{}}\toprule
& \multicolumn{2}{c}{$SNR = 10$} & \phantom{ab}& \multicolumn{2}{c}{$SNR = 50$} &
\phantom{ab} & \multicolumn{2}{c}{$SNR = \infty$}\\
\cmidrule{2-3} \cmidrule{5-6} \cmidrule{8-9}
$Regression Models$ & $RMSE$ & $RMDSE$ && $RMSE$ & $RMDSE$ && $RMSE$ & $RMDSE$ \\ \midrule
$\chi^2$      & 232      & \bf{100}&& 235      & 120    && 232      & \bf{100} \\
 PPR-ICA      & 242      & 128        && 242      &  99    && 280      & 162 \\
 GA-RF        & 308      & 183        && 248      & 136    && \bf{167} & 135 \\
 GA-GBM       & 287      & 160        && 248      & 149    && 233      & 113 \\
 GA-SVR       & \bf{221} & 122        && 281      & 151    && 299      & 160 \\
 GA-NNET      & 283      & 192        && 264      & 114    && 326      & 212 \\
 GA-KNN       & 238      & 120        && \bf{232} & 137    && 219      & \bf{100}  \\
 GA-MARS      & 253      & 113        && 254      & \bf{95}&& 226      & 133 \\
 GA-KPLS      & 275      & 120        && 300      & 119    && 387      & 218 \\
\bottomrule
\end{tabular}
\caption {Cross-validation RMSE and RMDSE for the various regression
  models that predict $T_{eff}$ (K).}
\label{tab:model_TSD} 
% \end{center}
\end{table*}

Table \ref{tab:model_TSD} shows that the performance of classifiers
based on the full spectrum (or in a compressed version in the form of
ICA components) and the best classifier based on features derived from
limited spectral bands is equivalent. The bartlett test shows that the
variances are homogeneous with a Bartlett\textquoteright s K-squared
of 8.5 with 2 degrees of freedom and a p-value of0.01426. The
Flinger-Killen test shows that homokedascity is verified at the
p=0.005886 level. Finally, the F-ANOVA test clearly shows that there
is no significant difference between models. Thus, we conclude that
the quality of features from the two approaches are equivalent in
predictive performance.  The difference between the performances of
the best classifier ($GA-KNN$; best on average over SNR), the minimum
$\chi^2$ classifier, and the $PPR-ICA$ classifiers are not
statistically significant. The bartlett test shows that the variances
are homogeneous with a Bartlett\textquoteright s K-squared of 8.5 with
2 degrees of freedom and a p-value of0.01426. The Flinger-Killen test
shows that homokedascity is verified at the p=0.005886 level. Finally,
the F-ANOVA test clearly shows that there is no significant difference
between models. Thus, we conclude that the quality of features from
the two approaches are equivalent in predictive performance.  In any
case, it is evident that the RMSE is significantly above the grid
spacing in temperature. We interpret the small differences as an
indication that there is as much information spread over the entire
spectrum shape as can be distilled from a few spectral bands.

The comparison with the effective temperatures compiled by
\cite{cesetti} shows however some significant differences across
models when evaluated not by the RMSE/RMDSE, but by the average bias
(see Table \ref{tab:model_Tbias}). 

\begin{table*}\centering
\ra{1.3}
\begin{tabular}{@{}lrrr@{}}\toprule
& {$SNR = 10$} & {$SNR = 50$} & {$SNR = \infty$}\\ \midrule
$\chi^2 $            &  -77 &  -87  & -85 \\
$ICA+ppr$            & -104 & -55   & -130 \\
GA-RR                & -102 &  -39  & 170 \\
GA-RF                & -173 & -127  &  -5 \\
GA-GBM               & -141 & -109  &  32 \\
GA-SVR               &  -58  &  -3  &  92 \\
GA-NNET              & -147 &  -36  &  39 \\
GA-KNN               &  -76  &-110  & -67 \\
GA-MARS              &  -57  & -88  &  98 \\
GA-KPLS              & -120 &   -4  & 214 \\
\bottomrule
\end{tabular}
\caption {Average bias in the $T_{eff}$ (K) estimates computed with
  respect to the reference values in Table 3 of \cite{cesetti}.}
\label{tab:model_Tbias} 
% \end{center}
\end{table*}

In general, all classifiers tend to predict lower effective
temperatures than those in the literature except in the noiseless
scenario. The models trained with noiseless spectra tend to
overestimate $T_{\rm eff}$, suggesting that the optimal SNR is between
SNR=50 and $\infty$. The minimum-$\chi^2$ approach and the GA-KNN
model systematically underestimate $T_{\rm eff}$ for all SNR
regimes. This shared behaviour is not surprising since minimum
$\chi^2$ is a single nearest neighbour method applied in the space of
the entire spectrum as opposed to the space selected features.

We have found in previous studies that, at least for input spaces
constructed from ICA compressions of the spectra, it is not necessary
to adapt the training set SNR to match exactly that of the prediction
set. On the contrary, we find that two regimes are sufficient to
obtain acceptable results. The two regimes are separated at
SNR=10. The model trained with SNR=50 spectra gives close to optimal
results for spectra with SNRs above 10, while below that limit the
same situation holds for the model trained with SNR=10 spectra. {\bf
  Cite paper by Ana.}

Figure~\ref{fig:irtf-teff} shows the correlation between the $T_{\rm
eff}$ estimates of the best (in the RMDSE sense) regression models and
the effective temperatures in Table 3 of \cite{cesetti}. 

%\begin {figure}
% \centering
%  \includegraphics[width=11cm]{figs/irtf-teff.pdf}
%  \caption{}
% \label{fig:irtf-teff}
%\end {figure}
 
%%%%%%%%%%%%%%%%%%%%%%%%%%%%%%%%%%%%%%%%%%%%%%%%%%%%%%%%%%%%%%
% Comparison with Teffs from spectral types.
%%%%%%%%%%%%%%%%%%%%%%%%%%%%%%%%%%%%%%%%%%%%%%%%%%%%%%%%%%%%%%

We then compare the predicted effective temperatures with the spectral
types listed in the IRTF spectral library in order to increase the
size of the validation sample beyond the 57 cases with estimated
temperatures in Table 3 of \cite{cesetti}. We converted the spectral
types into effective temperatures using the calibration of
\cite{2009ApJ...702..154S}. Both the RMSE and RMDSE were used to
evaluate the prediction accuracy (see Table~\ref{tab:model_Tvar}).

{\bf Faltan las tablas y figuras.}



%%%%%%%%%%%%%%%%%%%%%%%%%%%%%%%%%%%%%%%%%%%%%%%%%%%%%%%%%%%%%%
% TBD: Comparison with temperatures estimated with Cesetti features
%%%%%%%%%%%%%%%%%%%%%%%%%%%%%%%%%%%%%%%%%%%%%%%%%%%%%%%%%%%%%%

We have trained the same non linear regression models discussed above
using the features suggested by \cite{cesetti}. The performace of the
models based on these features are included in Table
\ref{tab:tab_CS_Model}.

\begin{table*}\centering
\ra{1.3}
\begin{tabular}{@{}rrrcrrcrr@{}}\toprule
& \multicolumn{2}{c}{$SNR = 10$} & \phantom{ab}& \multicolumn{2}{c}{$SNR = 50$} &
\phantom{ab} & \multicolumn{2}{c}{$SNR = \infty$}\\
\cmidrule{2-3} \cmidrule{5-6} \cmidrule{8-9}
$Regression Models$ & $RMSE$ & $RMDSE$ && $RMSE$ & $RMDSE$ && $RMSE$ & $RMDSE$ \\ \midrule
CS-RF               & 234       & 180       && {\bf 264} & 218       &&  {\bf 321} & 265 \\
CS-GBM              & {\bf 232} & 195       && 268       & 254       &&  325       & 246 \\
CS-SVR              & 268       & 227       && 293       & 257       &&  432       & 364 \\
CS-NNET             & 357       & 255       && 357       & {\bf 204} &&  552       & 435 \\
CS-KNN              & 249       & 172       && 293       & 256       &&  327       & {\bf 230}\\
CS-KPLS             & 351       & {\bf 162} && 856       & 456       && 1086       & 535 \\
\hline
\end{tabular}
\caption {Regression model performance based on the features proposed by \cite{cesetti}} 
\label{tab:tab_CS_Model}
%\end{center}
\end{table*}

{\bf How do you explain that the best SNR=10 model has the poorest
  performances for SNR=50 or $\infty$?}

From the comparison of Tables \ref{tab:tab_CS_Model} and
\ref{tab:model_TSD} we can draw the following conclusions:

\begin{itemize}
\item the RMSE for SNR=10 and 50 is equivalent for the regression
  models trained on GA features and those recommended in
  \cite{cesetti};
  \item however, the RMDSE is significantly higher in the case of the
    latter features for all SNR values.
    \item in the unrealistic case of noiseless spectra, the features
      proposed by \cite{cesetti} produce RMSE and RMDSE significantly
      worse than the GA features.
\end{itemize}

As a summary, we believe that the features found by the GA are to be
prefered to the ones proposed by \cite{cesetti}.

\subsubsection{Surface gravity models}

For the validation of our models, we only have 10 literature values of
the surface gravity available in Table 3 of
\cite{cesetti}. Unfortunately, this is too small a number to draw
significant conclusions on the comparison of methodologies from
external data. Hence, we are left only with plausibility arguments for
the selection of models. In this Section we will use $\log(T_{\rm
  eff})--\log(g)$ diagram comparisons to select the most plausible
model results. An important difference with respect to the models
discussed above is that we use the $T_{\rm eff}$ estimated in the
previous stage as input of our models. {\bf do we have some hint
  whether this was beneficial, neutral or detrimental?}

Table~\ref{tab:models_G_rmse} shows the RMSE and RMDSE of the
$\log(g)$ regression models for the same SNR regimes discussed for the
estimation of $T_{\rm eff}$.

\ra{1.3}
\begin{table*}\centering
\begin{tabular}{@{}rrrcrrcrr@{}}\toprule
& \multicolumn{2}{c}{$SNR = 10$} & \phantom{ab}& \multicolumn{2}{c}{$SNR = 50$} &
\phantom{ab} & \multicolumn{2}{c}{$SNR = \infty$}\\
\cmidrule{2-3} \cmidrule{5-6} \cmidrule{8-9}
$Regression Models$ & $RMSE$ & $RMDSE$ && $RMSE$ & $RMDSE$     && $RMSE$       & $RMDSE$ \\ \midrule
$\chi^2$          & 0.82       & 0.45      && 0.93       & 0.61       && 3.5        & 3.48 \\
$ PPR-ICA$        & 0.54       & 0.48      && {\bf 0.3}  & {\bf 0.17} && 0.72       & 0.57 \\
GA-RF             & 0.64       & \bf{0.38} && 0.77       & 0.72       && 0.53       & 0.39 \\
GA-GBM            & {\bf 0.48} & 0.45      && 0.61       & 0.47       && 0.49       & 0.41 \\
GA-SVR            & 0.66       & 0.40      && 0.63       & 0.58       && {\bf 0.46} & \bf{0.21} \\
GA-NNET           & 0.78       & 0.61      && 0.47       & 0.44       && 1.2        & 0.97 \\
GA-MARS           & 0.84       & 0.57      && 0.54       & 0.37       && 0.99       & 0.76 \\
GA-KNN            & 1.23       & 0.83      && 1.39       & 1.44       && 1.60       & 1.32 \\
GA-KPLS           & 0.99       & 0.99      && 0.51       & 0.49       && 0.96       & 0.77 \\
GA-RR             & 0.74       & 0.57      && 0.50       & 0.47       && 0.57       & 0.41 \\

\bottomrule
\end{tabular}
\caption {RMSE and RMDSE for the various $\log(g)$ regression models
  [dex].}
\label{tab:models_G_rmse} 
% \end{center}
\end{table*}

Again, as in the case of the effective temperatures, the differences
between the various models as measured by the RMSE or RMDSE are not
statistically significant.  This is not surprising given the
extraordinarily small sample of gravity measurements gathered from the
literature and used as reference for the computation of errors.
However, we can evaluate the models according to plausibility
arguments relative to the distribution of the model predictions in
$T_{\rm eff}$--$\log(g)$ diagrams.  Figure~\ref{fig:lt_lg_ga} shows
this distribution for four models selected based on these plausibility
criteria: GA-RR, GA-PLS, GA-KNN (the three of them for SNR=50), and
PPR-ICA (clockwise, starting at the top left corner).

\begin{figure}
 \begin{center}
   \includegraphics[width=\textwidth]{figs/ordieres-fig4.pdf}
 \caption{$\log(T_{eff})$--$\log(g)$ diagrams produced by the GA-KNN
   (SNR=$\infty$) effective temperatures and gravities derived with
   the GA-RR (SNR=50), GA-PLS (SNR=50), GA-NNR (SNR=50), and $\chi^2$ models (clockwise, starting from
   the top left plot).}
 \label{fig:lt_lg_ga}
 \end{center}
\end{figure}

{\bf Is $\chi^2$ much worse now for the weak parameter logg? I guess
  no. This needs be discussed}

{\bf Discuss these plots in the case of Cesetti features.}

\subsubsection{Metallicity models} 

Finally, the same machine learning models are trained to infer the
metallicity, again considering the effective temperature as an input
feature as in the $\log(g)$ regression
models. Table~\ref{tab:models_M_rmse} shows the RMSE and RMDSE
obtained for each regression model for the only seven M-type stars in
Table 3 of \cite{cesetti} with a metallicity estimate in the
literature.
%
% Metalicidad teórica desde Cesseti para las IRTF
%
\ra{1.3}
\begin{table*}\centering
\begin{tabular}{@{}rrrcrrcrr@{}}\toprule
& \multicolumn{2}{c}{$SNR = 10$} & \phantom{ab}& \multicolumn{2}{c}{$SNR = 50$} &
\phantom{ab} & \multicolumn{2}{c}{$SNR = \infty$}\\
\cmidrule{2-3} \cmidrule{5-6} \cmidrule{8-9}
$Regression Models$ & $RMSE$ & $RMDSE$ && $RMSE$ & $RMDSE$     && $RMSE$       & $RMDSE$ \\ \midrule
$\chi^2$    & 0.76 & 0.22      && 0.36 & 0.18     && 0.36 & 0.18 \\
$PPR-ICA$   & 0.24 & \bf{0.13} && 0.31 & 0.22     && 0.43 & 0.27 \\
$GA-RF$     & 0.33 & 0.25      && 0.73 & 0.41     && 0.61 & 0.36 \\
$GA-GBM$    & 0.27 & 0.19      && 0.70 & 0.52     && 0.63 & 0.35 \\
$GA-SVR$    & 0.33 & 0.22      && 0.45 & 0.32     && 0.92 & 0.89 \\
$GA-NNET$   & 0.37 & 0.30      && 0.33 & 0.37     && 0.95 & 0.81 \\
$GA-KNN$    & 0.69 & 0.55      && 0.23 & \bf{0.15}&& 0.21 & \bf{0.15} \\ 
$GA-MARS$   & 0.36 & 0.16      && 0.49 & 0.41     && 0.83 & 0.85 \\
$GA-RR$     & 0.31 & 0.17      && 0.30 & 0.24     && 0.78 & 0.23 \\

\bottomrule
\end{tabular}
\caption {RMSE and RMDSE for the various regression models predicting
  metallicity [dex].}
\label{tab:models_M_rmse} 
% \end{center}
\end{table*}

{\bf Compare the 7 or 6 values available. Discuss.  $\chi^2$ is
  the most popular method by far. We compare predictions of machine
  learning methods with minimum chi-squared. We first do histogram
  plots. Then, the same logTeff-logg plots as above but with
  metallicity coded in colour.}

% To be corrected
 Figure~\ref{M_ICA_10} shows the relationships between metalicity
 predicted by global espectrum estimation and GA feature based
 estimation against the real values provided by
 \cite{2013A&A...549A.129C} can be observed.

 \begin {figure}
  \centering
   \includegraphics[width=0.4\textwidth]{figs/irtf-figs/M-ICA10.pdf}
   \caption{Comparison between metallicity estimates from the
     literature and predictions from the PPR-ICA (SNR=10) model. {\bf
       TBC: Include description of symbols and colours.}}
  \label{M_ICA_10}
 \end {figure}

 {\bf Include table as annex with metallicities from the literature.}


\section{Physical parameters of the IPAC collection of spectra.}
%\input{ipac}
\subsection{Spectral bands selected}

As for the IRTF spectra, the spectral resolution of the BT-Settl
library was degraded to match the average resolution of IPAC spectra
in the Dwarf
Archives\footnote{http://spider.ipac.caltech.edu/staff/davy/ARCHIVE/index.shtml}. {\bf
What is the average resolution?}. Then, the spectra were trimmed to
produce valid segments between *** and *** {\AA}, which is the
spectral range common to all M stars in the archive. Finally, all
spectra were divided by the total integrated flux in this range in
order to factor out the stellar distance.

There is little hope {\it a priori} for reasonable accuracies with
regression models that predict the surface gravity and metallicity
from such wavelength-limited, low/intermediate resolution
spectra. Anyhow, we provide the results obtained applying the same
methodology as in Section \ref{irtf} to show the limitations.

\subsubsection{Spectral features for the estimation of effective temperatures.}

The application of the GA to the selection of features for the
prediction of effective temperature from noiseless spectra within the
IPAC wavelength range and resolution, results in the features included
in Table~\ref{tab:tab_NC_T}. Features are ordered by the fitness value
(the AIC){\bf and we only consider features that are present in at least 5
sets}.

\begin{table}
\begin{center}
\begin{tabular}{rrrr}
  \hline
  $\lambda_1$ & $\lambda_2$ & $\lambda_{cont;1}$ & $\lambda_{cont;2} $ \\ 
  \hline 
  
7062 & 7094.4 &	7314 & 7346.4 \\
7116 & 7148.4 &	7782 & 7814.4 \\
7134 & 7166.4 &	7872 & 7904.4 \\
6900 & 6932.4 &	7764 & 7796.4 \\
7170 & 7202.4 &	7890 & 7922.4 \\
7080 & 7112.4 &	7926 & 7958.4 \\
7188 & 7220.4 &	7548 & 7580.4 \\
7800 & 7832.4 &	7962 & 7994.4 \\
6990 & 7022.4 &	7008 & 7040.4 \\
7026 & 7058.4 &	6990 & 7022.4 \\

\hline
\end{tabular}
\caption {Spectral features and continuum bandpasses selected by the GA for
predicting $T_{\rm eff}$ using noiseless BT\_Settl
spectra.} \label{tab:tab_NC_T}
\end{center}
\end{table}

{\bf TBD by Luis: interpret the features.}

When noise is added to the BT-Settl spectra, we obtain the following
features depending on the SNR of the spectra:

\begin{table*}
\begin{center}
\begin{tabular}{rrrr | rrrr}
  \hline
 \multicolumn{4}{c}{SNR = 10} &  \multicolumn{4}{c}{SNR=50} \\
  \hline
$\lambda_1$ & $\lambda_2$ & $\lambda_{cont;1}$ & $\lambda_{cont;2} $ & $\lambda_1$ & $\lambda_2$ & $\lambda_{cont;1}$ & $\lambda_{cont;2} $ \\ 
  \hline
7692 & 7724.4 	6936 & 6968.4  & 7062 & 7094.4 &  7296 & 7328.4 \\
6990 & 7022.4 	7998 & 8030.4  & 7026 & 7058.4 &  7044 & 7076.4 \\
6900 & 6932.4 	7548 & 7580.4  & 7080 & 7112.4 &  7926 & 7958.4 \\
7854 & 7886.4 	7710 & 7742.4  & 6900 & 6932.4 &  7548 & 7580.4 \\
7116 & 7148.4 	7908 & 7940.4  & 7134 & 7166.4 &  7836 & 7868.4 \\
7278 & 7310.4 	7926 & 7958.4  & 7296 & 7328.4 &  7962 & 7994.4 \\
7152 & 7184.4 	7746 & 7778.4  & 6936 & 6968.4 &  7728 & 7760.4 \\
7134 & 7166.4 	7764 & 7796.4  & 6972 & 7004.4 &  6900 & 6932.4 \\
6918 & 6950.4 	6900 & 6932.4  & 6990 & 7022.4 &  7944 & 7976.4 \\
7224 & 7256.4 	7962 & 7994.4  & 6918 & 6950.4 &  7782 & 7814.4 \\

\hline
\end{tabular}
\caption {Spectral features and continuum bandpasses selected by the GA for predicting $ T_{eff}$ 
using BT\_Settl spectra with SNR=10 and 50.} \label{tab:tab_SNR1050_T}
\end{center}
\end{table*}

Tables \ref{tab:tab_SNRoo_G} and \ref{tab:tab_SNR1050_G} show the
spectral features selected by the GA for noiseless BT-Settl spectra
and the same spectra with SNR=10 and 50, respectively.

\begin{table}
\begin{center}
\begin{tabular}{rrrr}
  \hline
  $\lambda_1$ & $\lambda_2$ & $\lambda_{cont;1}$ & $\lambda_{cont;2} $ \\ 
  \hline

7134 & 7166.4 &	7044 & 7076.4 \\
6954 & 6986.4 &	7152 & 7184.4 \\
7512 & 7544.4 &	7890 & 7922.4 \\
7062 & 7094.4 &	7224 & 7256.4 \\
6936 & 6968.4 &	7854 & 7886.4 \\
6900 & 6932.4 &	7746 & 7778.4 \\
6918 & 6950.4 &	7800 & 7832.4 \\
7008 & 7040.4 &	7134 & 7166.4 \\
7872 & 7904.4 &	7008 & 7040.4 \\
7962 & 7994.4 &	7980 & 8012.4 \\

\hline
\end{tabular}
\caption {Spectral features and continuum bandpasses selected by the GA for predicting $\log(g)$ 
using noiseless BT\_Settl spectra.} \label{tab:tab_SNRoo_G}
\end{center}
\end{table}

\begin{table*}
\begin{center}
\begin{tabular}{rrrr | rrrr}
  \hline
 \multicolumn{4}{c}{SNR = 10} &  \multicolumn{4}{c}{SNR=50} \\
  \hline
$\lambda_1$ & $\lambda_2$ & $\lambda_{cont;1}$ & $\lambda_{cont;2} $ & $\lambda_1$ & $\lambda_2$ & $\lambda_{cont;1}$ & $\lambda_{cont;2} $ \\ 
  \hline

6990 & 7022.4 &	6918 & 6950.4 & 6918 & 6950.4 & 6936 & 6968.4  \\
6900 & 6932.4 &	7278 & 7310.4 & 6936 & 6968.4 & 7836 & 7868.4  \\
7062 & 7094.4 &	7242 & 7274.4 & 7656 & 7688.4 & 7890 & 7922.4  \\
7692 & 7724.4 &	7008 & 7040.4 & 6900 & 6932.4 & 7872 & 7904.4  \\
7656 & 7688.4 &	7998 & 8030.4 & 7008 & 7040.4 & 7044 & 7076.4  \\
6936 & 6968.4 &	7836 & 7868.4 & 7512 & 7544.4 & 7656 & 7688.4  \\
7206 & 7238.4 &	7062 & 7094.4 & 7440 & 7472.4 & 7332 & 7364.4  \\
7512 & 7544.4 &	7926 & 7958.4 & 7800 & 7832.4 & 7692 & 7724.4  \\
7764 & 7796.4 &	7710 & 7742.4 & 7404 & 7436.4 & 7548 & 7580.4  \\
7404 & 7436.4 &	7548 & 7580.4 & 7080 & 7112.4 & 7152 & 7184.4  \\
   \hline
\end{tabular}
\caption {Spectral features and continuum bandpasses selected by the GA for predicting $\log(g)$ 
using BT\_Settl spectra of SNR=10 and 50.} \label{tab:tab_SNR1050_G}
\end{center}
\end{table*}


Finally, the best features found by the GA for the estimation of the
metallicity are listed in Table~\ref{tab:tab_SNRoo_M} for the
noiseless BT-Settl spectra, and in Table~\ref{tab:tab_SNR1050_M} for
signal-to-noise ratios equal to 10 and 50.

\begin{table}
\begin{center}
\begin{tabular}{rrrr}
  \hline
  $\lambda_1$ & $\lambda_2$ & $\lambda_{cont;1}$ & $\lambda_{cont;2} $ \\ 
  \hline
7188 & 7220.4 &	7854 & 7886.4 \\ 
7080 & 7112.4 &	7926 & 7958.4 \\
7116 & 7148.4 &	7098 & 7130.4 \\
7422 & 7454.4 &	7836 & 7868.4 \\
7350 & 7382.4 &	7998 & 8030.4 \\
7224 & 7256.4 &	7818 & 7850.4 \\
7710 & 7742.4 &	7062 & 7094.4 \\
7476 & 7508.4 &	7944 & 7976.4 \\
7134 & 7166.4 &	7584 & 7616.4 \\
7836 & 7868.4 &	7278 & 7310.4 \\
\hline
\end{tabular}
\caption {Spectral features and continuum bandpasses
selected by the GA for predicting metallicity using noiseless
BT\_Settl spectra.} \label{tab:tab_SNRoo_M}
\end{center}
\end{table}

\begin{table*}
\begin{center}
\begin{tabular}{rrrr | rrrr}
  \hline
 \multicolumn{4}{c}{SNR = 10} &  \multicolumn{4}{c}{SNR=50} \\
  \hline
$\lambda_1$ & $\lambda_2$ & $\lambda_{cont;1}$ & $\lambda_{cont;2} $ & $\lambda_1$ & $\lambda_2$ & $\lambda_{cont;1}$ & $\lambda_{cont;2} $ \\ 
  \hline
7692 & 7724.4 &	7026 & 7058.4  &  7098 & 7130.4 & 7926 & 7958.4 \\
6900 & 6932.4 &	7008 & 7040.4  &  7188 & 7220.4 & 7962 & 7994.4  \\
7350 & 7382.4 &	7908 & 7940.4  &  7368 & 7400.4 & 7980 & 8012.4  \\
6918 & 6950.4 &	6900 & 6932.4  &  7116 & 7148.4 & 7872 & 7904.4  \\
7098 & 7130.4 &	7314 & 7346.4  &  7062 & 7094.4 & 7206 & 7238.4  \\
7440 & 7472.4 &	7872 & 7904.4  &  7584 & 7616.4 & 7170 & 7202.4  \\
7134 & 7166.4 &	7962 & 7994.4  &  6936 & 6968.4 & 6918 & 6950.4  \\
7368 & 7400.4 &	7926 & 7958.4  &  7692 & 7724.4 & 7890 & 7922.4  \\
7080 & 7112.4 &	7044 & 7076.4  &  7134 & 7166.4 & 7548 & 7580.4  \\
7044 & 7076.4 &	7980 & 8012.4  &  7494 & 7526.4 & 7998 & 8030.4  \\
\hline
\end{tabular}
\caption {Spectral features and continuum bandpasses
selected by the GA for predicting metallicities using BT\_Settl
spectra of SNR=10 and 50.} \label{tab:tab_SNR1050_G}
\end{center}
\end{table*}


\subsection{Regression models}
%
In the following, we will summarise the results obtained for the IPAC
data set. We deal with the different physical paramenters in separate
Sections. We start by reporting the cross validation Root Mean Square
Errors (RMSE) and Root Median Square Error (RMDSE )for the five-fold
cross-validation strategy, and we subsequently discuss the accuracy of
the predictions with respect to literature values where available.

\subsubsection{Effective temperature models}

Table \ref{tab:model_TSD_IPAC} summarises the RMSE/RMDSE for the
complete set of models: the minimum $\chi^2$ estimate based on the
full spectrum ($\chi^2$), the projection pursuit regression based on
the ICA components (PPR-ICA) and some models trained on the spectral
features proposed by the GA (GA-RF, GA-GBM, GA-SVR, GA-NNET, GA-MARS,
GA-KPLS). For each model, we report the RMSE/RMDSE obtained for
several noise levels of the training sets.

%\newcommand{\ra}[1]{\renewcommand{\arraystretch}{#1}}
\begin{table*}\centering
\ra{1.3}
\begin{tabular}{@{}rrrcrrcrr@{}}\toprule
& \multicolumn{2}{c}{$SNR = 10$} & \phantom{ab}& \multicolumn{2}{c}{$SNR = 50$} &
\phantom{ab} & \multicolumn{2}{c}{$SNR = \infty$}\\
\cmidrule{2-3} \cmidrule{5-6} \cmidrule{8-9}
$Regression Models$ & $RMSE$ & $RMDSE$ && $RMSE$ & $RMDSE$ && $RMSE$ & $RMDSE$ \\ \midrule
$\chi^2$    & {\bf 147} & 79       && {\bf 121} & {\bf 56}  && {\bf 126} & {\bf 57} \\
$ PPR-ICA$  & 188       & 126      && 164       & 95        && 191       & 130 \\
GA-RF       & 160       & 97       && 196       & 103       && 145       & 94 \\
GA-GBM      & 175       & 105      && 225       & 99        && 185       & 94 \\
GA-SVR      & 203       & 112      && 285       & 106       && 368       & 154 \\
GA-NNET     & 221       & 84       && 313       & 111       && 395       & 202 \\
GA-KNN      & 183       & 119      && 193       & 109       && 224       & 110  \\
GA-MARS     & 222       & 76       && 361       & 103       && 374       & 157 \\
GA-KPLS     & 227       & {\bf 72} && 331       & 123       && 409       & 208 \\
\bottomrule
\end{tabular}
\caption {RMSE and RMDSE for the various regression models that predict $T_{eff}$ (K).} 
\label{tab:model_TSD_IPAC} 
% \end{center}
\end{table*}

Again, as in the IRTF case, we see that the compression of the spectra
results in a performance degradation. We believe that this is due to
the information being spread over the entire spectrum rather than
concentrated in a few bands. {\bf What about the curse of
  dimensionality? }


%%%%%%%%%%%%%%%%%%%%%%%%%%%%%%%%%%%%%%%%%%%%%%%%%%%%%%%%%%%%%%%
% Comparison with the Teff from SpType calibration.
% Because for IPAC we do not have Teff estimates?
%%%%%%%%%%%%%%%%%%%%%%%%%%%%%%%%%%%%%%%%%%%%%%%%%%%%%%%%%%%%%%%


{\bf Explain the spt-teff calibration used.}

{\bf Biases?}

{\bf We do have problems with the prediction at low temperatures when
trained with SNR= 10 or 50.}

{\bf Include plot with 4 models}

\begin {figure*}
 \centering
 \begin{subfigure}{.35\textwidth}
  \centering
  \includegraphics[width=11cm]{figs/ipac_T_ICAoo_LSB.pdf}
  \caption{Comparison between Temperature estimations from Theoretical Temperature 
  in x axis and the modeled ICA based estimation at SNR=$\infty$ on y-axis}
 \label{fig:ipac_icaoo_lsb}
 \end{subfigure}
  \begin{subfigure}{.35\textwidth}
  \centering
  \includegraphics[width=11cm]{figs/ipac_T_RFoo_LSB.pdf}
  \caption{Comparison between Temperature estimations from Theoretical Temperature 
  in x axis and the featured based Random Forest modeled at SNR=$\infty$ on y-axis}
 \label{fig:ipac_rfoo_lsb}
 \end{subfigure}
  \begin{subfigure}{.35\textwidth}
  \centering
  \includegraphics[width=11cm]{figs/ipac_T_RF50_LSB.pdf}
  \caption{Comparison between Temperature estimations from Theoretical Temperature 
  in x axis and the featured based Random Forest modeled at SNR=$50$ on y-axis}
 \label{fig:ipac_rf50_lsb}
 \end{subfigure}
 \label {fig:comp01}
 \caption{Performance comparison between the different strategies for Teperature prediction}
\end {figure*}

%%%%%%%%%%%%%%%%%%%%%%%%%%%%%%%%%%%%%%%%%%%%%%%%%%%%%%%%%%%%%%%
% Comparison with predictions from Cesetti's features. 
%%%%%%%%%%%%%%%%%%%%%%%%%%%%%%%%%%%%%%%%%%%%%%%%%%%%%%%%%%%%%%%

Having shown that the feature selection with GAs degrades the
performance of regression models, one can wonder whether a different
feature selection procedure would produce better results. In
particular, we investigate the possibility that the features proposed
by \cite{cesetti} result in a performance equal to or even better than
the one achieved with $\chi^2$.

%\begin{table}
%\begin{center}
%\begin{tabular}{rrrr}
%  \hline
%  $\lambda_1$ & $\lambda_2$ & $\lambda_{cont;1}$ & $\lambda_{cont;2} $ \\ 
%  \hline
%8461 & 8474 & 8474 & 8484 \\
%8484 & 8513 & 8474 & 8484 \\
%8522 &  8562 & 8474 & 8484 \\
%8577 & 8619 & 8563 & 8577 \\
%8642 & 8682 & 8619 & 8642 \\
%8730 & 8772 & 8700 & 8725 \\
%8802 & 8811 & 8776 & 8792 \\
%8850 & 8890 & 8815 & 8850 \\
%9000 & 9030 & 8983 & 8998 \\
%9080  & 9100 & 9040 & 9050 \\
%\hline
%\end{tabular}
%\caption {Features selected by following suggestions from Cesetti et al, table 1. } 
%\label{tab:tab_CS_T}
%\end{center}
%\end{table}

We train the same regression models applied to the GA selected
features, to the features selected in \cite{cesetti}, again learning
from BT-Settl spectra of various SNRs and predicting over the IPAC
set. A summary of the results can be found in Table
\ref{tab:tab_CS_Model}, where we use CS- to indicate that the model was
trained using the features by \cite{cesetti}.

\begin{table*}
\begin{center}
\begin{tabular}{@{}rrrcrrcrr@{}}\toprule
& \multicolumn{2}{c}{$SNR = 10$} & \phantom{ab}& \multicolumn{2}{c}{$SNR = 50$} &
\phantom{ab} & \multicolumn{2}{c}{$SNR = \infty$}\\
\cmidrule{2-3} \cmidrule{5-6} \cmidrule{8-9}
$Regression Models$ & $RMSE$ & $RMDSE$ && $RMSE$ & $RMDSE$     && $RMSE$       & $RMDSE$ \\ \midrule
CS-RF   & 203       & 140       && 243       & {\bf 121} &&  {\bf 306} &  {\bf 172}  \\
CS-GBM  & {\bf 188} & {\bf 120} && {\bf 161} & 138       &&  337       &  222  \\
CS-SVR  & 197       & 135       && 379       & 194       &&  840       &  688  \\
CS-NNET & 207       & 135       && 514       & 296       &&  719       &  489  \\
CS-MARS & 252       & 124       && 789       & 186       && 3464       &  784  \\
CS-KNN  & 235       & 158       && 246       & 137       &&  314       &  175  \\
CS-KPLS & 250       & 201       && 741       & 361       && 2247       & 1424  \\
CS-RR   & 211       & 128       && 400       & 239       &&  828       &  774  \\

\hline
\end{tabular}
\caption {Performances of regression models trained on the features
  selected by \cite{cesetti} applied to BT-Settl spectra.}
\label{tab:tab_CS_Model}
\end{center}
\end{table*}

For SNR=10, the GA best models (GA-KPLS in RMDSE or GA-RF in RMSE)
outperform the best CS model (GA-GBM). For SNR=50 the situation
depends on the figure-of-merit used to compare the classifiers: in
RMSE the best model is CS-GBM while in RMDSE GA-GBM outperforms all
CS-models. Finally, for the unrealistic case of noiseless spectra,
Table \ref{tab:tab_CS_Model} shows an overwhelming degradation of the
prediction accuracy from CS- features. {\bf Overfitting?} But even in
the only case where the CS features outperform those selected by the
GA, the performance is below the one achieved by the minimum-$\chi^2$
approach.

%%% HERE 1 %%%
The relationship between the GA predicted Temperature and the one
measured by Rojas-Ayala can be found in the
Figure~\ref{fig:ipac_lt_lt}
\begin{figure}
 \begin{center}
 \includegraphics[width=12cm]{figs/ipac_LG_Trojas_Tknn_10.pdf}
 \caption{Relationship between $log(T) from Rojas-Ayala $ in the x axis 
 and $log(T)$ as predicted by KNN with SNR=$10$}
 \label{fig:ipac_lt_lt}
 \end{center}
\end{figure}
%%%%%%%%%%%%%

\subsubsection{Surface gravity models}

As in the IRTF exercise, we attempt to select features for surface
gravity estimation from BT-Settl spectra using GAs despite the much
lower spectral resolution and smaller wavelength coverage of the IPAC
spectra. Since there is no substantive compilation of surface
gravities that we could cross match with the IPAC list of M stars in
the Dwarf Archive, we are left with the same plausibility arguments
used in the IRTF study which are based on the $\log(T_{\rm
  eff})$--$\log(g)$ diagram.

We again use the effective temperatures as input of the regression
models. Table~\ref{tab:models_G_rmse} shows the cross-validation RMSE
and RMDSE for the same set of regression models used throughout this
article. It shows that the GA-RF model outperforms all other in all
SNR regimes, giving a consistent RMDSE of 1.0 dex. Obviously, this is
barely enough for classification in luminosity classes.

% HERE 2: This must be wrong: the Chi^2 and ICA columns are almost empty
% so I have asked Joaquín why. If the RMDSE are computed from these empty columns, these figures are just wrong.

\ra{1.3}
\begin{table*}\centering
\begin{tabular}{@{}rrrcrrcrr@{}}\toprule
& \multicolumn{2}{c}{$SNR = 10$} & \phantom{ab}& \multicolumn{2}{c}{$SNR = 50$} &
\phantom{ab} & \multicolumn{2}{c}{$SNR = \infty$}\\
\cmidrule{2-3} \cmidrule{5-6} \cmidrule{8-9}
$Regression Models$ & $RMSE$ & $RMDSE$ && $RMSE$ & $RMDSE$     && $RMSE$       & $RMDSE$ \\ \midrule
$\chi^2$    & 2.2       & 1.6       && 2.2       & 1.4       && 2.2       & 1.6 \\
PPR-ICA     & 2.1       & 1.8       && 1.8       & 1.4       && 4.3       & 4.2 \\
GA-RF       & {\bf 1.3} & {\bf 1.0} && {\bf 1.6} & {\bf 1.1} && {\bf 1.4} & {\bf 0.9} \\
GA-GBM      & 1.6       & 1.1       && 1.7       & 1.4       && 1.7       & 1.2 \\
GA-SVR      & 2.0       & 1.8       && 2.1       & 1.9       && 2.3       & 1.6 \\
GA-NNET     & 2.0       & 1.8       && 2.2       & 1.9       && 3.2       & 2.8 \\
GA-MARS     & 1.8       & 1.5       && 2.0       & 1.7       && 2.0       & 1.5 \\
GA-KNN      & 2.0       & 1.5       && 2.2       & 1.7       && 1.7       & 1.2 \\
GA-KPLS     & 1.8       & 1.4       && 2.0       & 1.7       && 2.7       & 2.3 \\
GA-RR       & 2.0       & 1.8       && 2.1       & 1.8       && 3.7       & 3.2 \\

\bottomrule
\end{tabular}
\caption {RMSE and RMDSE for the various regression models predicting $Log(G)$ [dex].} 
\label{tab:models_G_rmse} 
% \end{center}
\end{table*}

Figure \ref{fig:teffvsloggIPAC} shows the $\log(T_{\rm
  eff})$--$\log(g)$ diagram for the GA-RF and GA-NNET models. The
latter is, in our opinion, the one that shows the diagram that is most
with Fig. \ref{} in this work, and Fig. 1 in \cite{cesetti}. All GA-
models predict decreasing surface gravities for main sequence stars
below $\log(T_{\rm eff}=3.6$. GA-NNET predicts main sequence values
between $4 \le \log(g) \le6$, while luminosity classes III-I appear
clearly separated from the main sequence with values concentrated in
the 4-6 range except for the hottest cases with $\log(T_{rm eff} >
3.55$. The GA-RF results, despite showing the best cross-validation
errors (RMSE/RMDSE), result in unrealistic main sequence gravities. We
interpret this as the result of overfitting to the training examples. 

\begin{figure*}
 \begin{center}
 \includegraphics[width=12cm]{figs/ipac-teff-logg.pdf}
 \caption{Relationship between $log(T) $ ($x$ axis) 
 and $log(g)$ ($y$ axis) for several regression models.}
 \label{fig:teffvsloggIPAC}
 \end{center}
\end{figure*}

{\bf Right now, it appears that feature selected models are worse than $\chi^2$, judging only from the 10 available estimates (mail sent to jbmere). If so, the conclusion is clear: we should not do feature selection at these resolutions. This is useful as Cesseti et al do not question the utility of feature selection. For the IRTF (which is the dataset used by Cesseti et al), we should check this: are the models with feature selection better than $\chi^2$?}

\subsubsection{Metallicity models} 

Finally, the same analysis is performed for the Metalicty parameter, 
again by considering Temperature as a fixed feature.
In Table~\ref{tab:models_M_rmse} 
we can see the analysis of performance of different classes of
models and cosidering a variety in features. The checks were carried out against 
$Met$ from Neves III.

%
% Metalicidad teórica desde NevesIII para IPAC
%
\ra{1.3}
\begin{table*}\centering
\begin{tabular}{@{}rrrcrrcrr@{}}\toprule
& \multicolumn{2}{c}{$SNR = 10$} & \phantom{ab}& \multicolumn{2}{c}{$SNR = 50$} &
\phantom{ab} & \multicolumn{2}{c}{$SNR = \infty$}\\
\cmidrule{2-3} \cmidrule{5-6} \cmidrule{8-9}
$Regression Models$ & $RMSE$ & $RMDSE$ && $RMSE$ & $RMDSE$     && $RMSE$       & $RMDSE$ \\ \midrule
$\chi^2 BTSettl$    & 0.55    & 0.27   && 0.51 & 0.29 && 0.43  & 0.29 \\
$ ICA+ ppr$         & 0.48 & 0.27 && 0.70  & 0.39 && 0.85  & 0.71 \\
$rf $               & 0.55 & 0.38    && 0.71  & 0.61   && 0.23  & 0.16 \\
$gbm $              & 0.64 & 0.43 && 0.87  & 0.84  && 0.31  & 0.23 \\
$ svr $             & 0.46 & 0.26   && 0.57 & 0.44  && 3.38  & 2.33 \\
$ nnet $            & 0.52 & 0.45      && 0.66 & 0.54  && 2.03  & 1.88 \\
$ knn $             & 0.37  & 0.28   && 0.99  & 0.78 && 0.56 & 0.32 \\ 
$ mars+ bagging $   & 0.71  & 0.47 && 0.80   & 0.69   && 1.15    & 0.68 \\
$ pls $             & 0.67  & 0.61  && 0.63  & 0.55 && 1.17 & 1.02 \\ 
$Rule Regression $  & 0.47 & 0.29 && 0.50 & 0.36  && 1.18 &  1.18 \\

\bottomrule
\end{tabular}
\caption {RMSE and RMDSE for the various regression models predicting $Met$ [dex].} 
\label{tab:models_M_rmse} 
% \end{center}
\end{table*}


{\bf Noooo}
The relationship between the GA predicted Temperature and the one measured by Rojas-Ayala can be 
found in the Figure~\ref{fig:ipac_mt}
\begin{figure}
 \begin{center}
 \includegraphics[width=12cm]{figs/ipac_Met_10_NevesIII.pdf}
 \caption{Relationship between $ T from NevesIII $ in the x axis 
 and $ Met $ as predicted by Regression Rules with SNR=$10$}
 \label{fig:ipac_mt}
 \end{center}
\end{figure}


% In Figure~\ref{fig:M_chi2_50_cesetti} and Figure~\ref{fig:M_GAM_1010_Cesetti} 
% relationships between metalicity predicted by global espectrum estimation 
% and GA feature based estimation against the real values
% provided by \cite{2013A&A...549A.129C} can be observed.

% \begin {figure}
%  \centering
%  \begin{subfigure}{.85\textwidth}
%   \centering
%   \includegraphics[width=12cm]{figs/M_Chi2_50_Cesetti.pdf}
%   \caption{Comparison between Metalicity estimations from Spectral Subtype 
%  in x axis and the closest BT\_Settl spectra by $\chi^2$ at SNR=$50$ on y-axis}
%  \label{M_chi2_50_cesetti}
%  \end{subfigure}
%   \begin{subfigure}{.85\textwidth}
%   \centering
%   \includegraphics[width=12cm]{figs/M_GAM_1010_Cesetti.pdf}
%   \caption{Comparison between Metalicity estimations from Spectral Subtype 
%  in x axis and the Support Vector Machines for Ga based features trained with BT\_Settl 
%  at SNR=$\infty$ and features for forecasting at SNR=$\infty$ on y-axis}
%  \label{fig:M_GAM_1010_Cesetti}
%  \end{subfigure}
%  \label {fig:comp03}
%  \caption{Performance comparison between the $chi^2$ based selection 
%           and the band oriented features to forecast Log(g)}
% \end {figure}
%
   
   

% De nuevo, el análisis y discusión, función de lo que queramos dejar



\section{Conclusions}

\begin{acknowledgements}
This research has benefitted from the M, L, T, and Y dwarf compendium housed at DwarfArchives.org.
The authors also thanks to the Spanish Ministery for Economy and Innovation because of the 
support obtained through the project with ID: AyA2011-24052. IRTF library provided by the 
University of Hawaii under Cooperative Agreement no. NNX-08AE38A with the National 
Aeronautics and Space Administration, Science Mission Directorate, Planetary Astronomy Program.
\end{acknowledgements}

%-------------------------------------------------------------------

\bibliography{references_m}{}
\bibliographystyle{bibtex/aa}

\end{document}
